% !TeX spellcheck = en_GB
\section{Modular forms}
\subsection{Modular forms of level 1}
Let $\halfplane$ denote the \textit{upper-half plane}, that is, 
$$\halfplane = \{z = x+yi \in \C | \ y>0 \}.$$
We say that a function $f:\halfplane 
 \C$ is \textit{weakly modular} of \textit{weight} $2k$ if $f$ is meromorphic and
$$
f(z) = (cz+d)^{-2k} f \left( \frac{az+b}{cz+d} \right)
\qquad \text{ for all }
\begin{pmatrix} a & b\\
				c & d
\end{pmatrix}
\in \SL2{\Z}.
$$
The group $\SL2{\Z}$ of invertible 2-by-2 matrices over $\Z$ with  is generated by
$$
S = \begin{pmatrix} 0 & -1 \\
					1 &  0
\end{pmatrix}
\quad \text{ and } \quad
T = \begin{pmatrix} 1 & 1 \\
					0 & 1 
\end{pmatrix};
$$
see \cite[p.1-2]{SL2Z}.
From this property, we can derive an alternative definition of weakly modular functions:
$f$ is weakly modular of weight $2k$ if $f$ is meromorphic and
$$
f(z+1) = f(z) \quad \text{ and } \quad f(-1/z) = z^k f(z),
$$
for all $z \in \C$.
Moreover, we define a function $f:\halfplane \mapsto \C$ to be \textit{modular} of weight $2k$ if $f$ is holomorphic and weakly modular.
Lastly, we say that a function $f:\halfplane \mapsto \C$ is a \textit{modular form} of weight $2k$ if it modular and holomorphic at $\infty$, that is, $f(\nicefrac{1}{z})$ is holomorphic at $z=0$.

It is straightforward to check, using the above definition, that the set of modular forms of weight $2k$ is closed under addition and multiplication by complex scalars.
More precisely:
\begin{itemize}
    \item If $f_1$ and $f_2$ are modular forms of weight $2k$, then $f_1+f_2: z \mapsto f_1(z)+f_2(z)$ is modular of weight $2k$ as well.
    
    \item Similarly, if $\lambda \in \C$ and $f$ is a modular form of weight $2k$, then so is $\lambda \cdot f: z \mapsto \lambda f(z)$.
\end{itemize}
Therefore, modular forms of weight $2k$ over $\C$ form a vector space. We denote it $M_k$.

It is also possible to multiply modular forms, in which case the weights are additive:
If $f_1$ and $f_2$ are modular forms of respective weights $2k_1$ and $2k_2$, then $f_1 \cdot f_2:z \mapsto f_1(z)f_2(z)$ is modular of weight $2k_1+2k_2$.

We deduce that we can take powers of modular forms, and the weight is then multiplied by the exponent:
if $f(z)$ is modular of weight $2k$, then $f^n(z)$ is modular of weight $2k \cdot n$ (with $n \in \N$ \footnote{The set of naturals $\N$ is taken to start from $0$ in this paper.} ).



\subsection{Typical Modular Forms}
\subsubsection{Eisenstein series $G_k$}
The most famous class of modular forms is probably the \textit{Eisenstein series}, usually denoted $G_k$. We define them as follows \cite[Examples of Modular Forms]{ModularFormsComputationalApproach}:
$$
G_k(z) = \sum_{(m,n) \in \Z^2\setminus\{(0,0)\}} \frac{1}{(mz+n)^{2k}}
$$
for $k \geq 2$.

It is easy to check that $G_k$ are modular of weight $2k$ \cite[Proposition 2.1]{ModularFormsComputationalApproach}, as:
$$
G_k(z+1) = G_k(z)
$$
(using $(m,n+m) \mapsto (m,n)$, an invertible map) and
$$
G_k(-1/z) = z^k G_k(z)
$$
(using $(m,-n) \mapsto (m,n)$, an invertible map).
It is pleasant to remark that \cite[Proposition 2.2]{ModularFormsComputationalApproach}
$$
G_k(\infty) = \sum_{n \in \Z \setminus \{0\} } \frac{1}{n^{2k}} = 2\zeta(2k),
$$ where $\zeta(k)$ is Riemann zeta function.
The values of this function are well-known on positive even numbers, and we deduce \cite[p.194]{MathHandbook} that:
$$
G_k(\infty) = 2\zeta(2k) = \frac{(2\pi)^{2k}}{(2k)!}B_k,
$$
where $B_k = (-1)^{k+1} b_{2k}$ and $b_k$ are Bernoulli numbers.

\subsubsection{The Modular Discriminant $\Delta$}
We will be interested in one main modular form in the rest of this article: the \textit{modular discriminant} $\Delta$.
We define $\Delta$ in terms of $G_k$ as follows \cite[p.84]{CourseInArithmetic}:
$$
\Delta = \left( \frac{1}{(2\pi)^{12}} \right) (g_2^3 - 27g_3^2) \in M_6 \qquad \text{ with } g_2 = 40G_2 \text{ and } g_3 = 140G_3
$$
As $g_2^3$ is modular of weight $4 \cdot 3=12$ and $g_3^2$ of weight $6 \cdot 2 = 12$, $\Delta$ is modular of weight $12$.
Multiplying by the scalar $\left( \nicefrac{1}{(2\pi)^{12}} \right)$ doesn't change the weight of the modular form, and it will we useful later for normalization purposes.

Now, using 
$
G_2(\infty) = 2\zeta(4) = \frac{\pi^4}{45}
$
 and 
$
G_3(\infty) = 2\zeta(6) = \frac{2\pi^4}{945}
$
, we get 
$$
\Delta(\infty) = \left( \frac{1}{(2\pi)^{12}} \right) \left[ \left( \frac{4\pi^4}{3} \right)^3 - \left( \frac{8\pi^4}{27} \right)^2 \right] =  0
$$
so $\Delta$ has a zero at infinity.


\subsection{Cusp Forms}
A function $f:\halfplane \to \C$ that is a modular form may in addition be a \textit{cusp form}, if $f(\infty)=0$.
We will denote the \textit{space of modular cusp forms} of weight $2k$ over $\C$ by $M_k^0$. This space of cusp forms of weight $2k$ is a subset of the space of modular forms of weight $2k$.

It is useful to note $G_k(\infty) = \sum_{n \in \N^*} \frac{2}{n^{2k}} > 2$ and in particular, $G_k(\infty) \neq 0$, so $G_k$ are \textit{not} cusp forms for any $k$.
As we have shown it before, $\Delta(\infty)=0$, so $\Delta$ is a modular cusp form of weight 12, so $\Delta \in M_6^0$.
Using tools from complex analysis, we can prove that $\Delta$ has only one zero (at infinity), which has order one \cite[p.88]{CourseInArithmetic}.

We have the following relation:
\begin{theorem}
	\cite[p.88]{CourseInArithmetic}:
    $M_k \cong M_k^0 \oplus \C \cdot G_k \quad \text{ for all } k \geq 2$
\end{theorem}
\begin{proof}
    We let $\Phi:M_k \to \C$ such that if $f \in M_k$, $\Phi(f) = f(\infty)$.
    Now, we have $\Ker{\Phi} = M_k^0$, therefore, by the 1\textsuperscript{st} Isomorphism Theorem, $\nicefrac{M_k}{M_k^0} \cong \Image{\Phi} \subseteq \C$.
    Note that $G_k \in M_k$, and $G_k(\infty) = \sum_{n \in \Z^*} \frac{1}{n^{2k}} \neq 0$, so $G_k \not\in M_k^0$.
    As $G_k \neq 0$, $\dim(\nicefrac{M_k}{M_k^0}) \geq 1$ and $\Image{\Phi} = \C$.
    Thus, $G_k \in M_k \setminus M_k^0$.
    
    Finally, we have $M_k \cong M_k^0 \oplus \C \cdot G_k$ if $k \geq 2$.
    (The above argument fails for $k<2$ as $G_k$ is not well defined any more.)
\end{proof}
Therefore, the dimensions of $M_k$ and $M_k^0$ are closely linked.



\subsection{Dimensions of Spaces of Modular Forms}
The fact that multiplying two modular forms gives a function that remains modular yields that we may map a set of modular forms to another.

\begin{theorem}\cite[p.88]{CourseInArithmetic}.
	We have $M_{k-6} \cong M_k^0$.
\end{theorem}
\begin{proof}
    We define $\Phi:M_{k-6} \to M_k^0$ by setting, 
    $$\Phi(f)(z) = \Delta(z)f(z) \quad \text{ for } f \in M_{k-6}.$$
    This is well defined as if $f$ has weight $2(k-6)$, $\Delta \cdot f$ has weight $2k$ since $\Delta$ has weight $12$. As $\Delta$ is a cusp from, $\Delta \cdot f$ will also be a cusp form.
    From the definition, $\Phi$ is a linear map.
    
    Now, if $g \in M_k^0$, we may define $\Psi: M_k^0 \to M_{k-6}$ by setting
    $$\Psi(g)(z) = \nicefrac{g(z)}{\Delta(z)} \quad \text{ for } g \in M_k^0.$$
    This is well defined as if $g$ has weight $2k$, $\Delta \cdot f$ has weight $2k$ since $\Delta$ has weight $12$.
    This is well defined as $\Delta$ has only one zero, at infinity, where $g$ also has a zero (as $g$ is a cusp form). The weights agree again as well.
    We remark that $\Psi = \Phi^{-1}$. So $\Phi$ is bijective, and thus an isomorphism. 
    Finally, we have $M_{k-6} \cong M_k^0$.
\end{proof}
This theorem, combined with the previous one is very powerful: it shows that there must be a pattern in the sequence of dimensions $\dim(M_k)$ and $\dim(M_k^0)$ for $k \geq 2$.
We have $M_k \cong M_k^0 \oplus \C \cdot G_k \cong M_{k-6} \oplus \C \cdot G_k$, so $\dim(M_k) = \dim(M_{k-6})+1$ when $k \geq 2$.
Thus, if we compute the dimensions of $M_0$, $M_1$, $M_2$, $M_3$, $M_4$, $M_5$, we can extrapolate dimensions of $M_k$ and $M_k^0$ for all $k$.

Using complex analysis techniques again, we have:
\begin{itemize}
    \item $\dim(M_k) = 0 \text{ for } k < 0$
    \item $\dim(M_1) = 0$
    \item $\dim(M_0) = \dim(M_2) = \dim(M_3) = \dim(M_4) = \dim(M_5) = 1$
\end{itemize}
%[proof?]
In the case $k=0$, $\dim(M_0) = 1$. As $f(z) = 1$ is clearly a modular from of weight $0$, $\{1\}$ is a basis for $M_0$. We deduce $\dim(M_k^0) = 0$ as $1$ is clearly not a cusp form.
In the case $k=1$, $\dim(M_1) = 0$, which makes $\dim(M_1^0) = 0$ automatically.
(Cases $k<0$ are similar to $k=1$.)

Other cases may be derived directly from the relations (using induction to get general formulas), and we obtain:

%[Table of dimensions of $M_k$ and $M_k^0$]
\begin{center}
\begin{tabular}{||c||c|c|c||} 
    \hline
    Space & $k<0$ & $k \geq 0, \ k \equiv 1 \bmod 6$ & $k \geq 0, \ k \not \equiv 1 \bmod 6$ \\
    \hline
    \hline
    $\dim(M_k)$ & $0$ & $\floor{\nicefrac{k}{6}}$ & $\floor{\nicefrac{k}{6}} + 1$ \\
    \hline
    $\dim(M_k^0)$ & $0$ & $\max\{0, \floor{\nicefrac{k}{6}} - 1\}$ & $\floor{\nicefrac{k}{6}}$ \\
    \hline
\end{tabular}
\end{center}
Note that the $\max$ is taken only to avoid negative dimensions.



\subsection{Fourier Expansion}
\subsubsection{Definition}
To study modular forms, we can use Fourier expansion.
As a modular form $f$ satisfies $f(z) = f(z+1)$ for all $z \in \C$, we can express the Fourier expansion in terms of $q = e^{2 \pi i z}$.
Thus, in the case of $f$ being a modular form of weight $2k$, 
a \textit{Fourier expansion} is a representation of $f$ as a power series of $e^{2\pi i n z}$
i.e. $$f(z) = \sum_{n \in \Z} a_n(f) e^{2\pi inz}.$$
We usually denote $q = e^{2\pi i z}$ so that $q^n = e^{2\pi i n z}$ 
and the Fourier expansion of $f$ become 
$$
f(q) = \sum_{n \in \Z} a_n(f) q^n.
$$
When in this form, we may as well call it the \textit{$q$-expansion}.

\paragraph{Asymptotic Notation}
The asymptotic behaviour of a function may be expressed in terms of another.
This is done via the \textit{big-O notation} or \textit{asymptotic notation}.
We write $f(x) = \mathcal{O}(g(x))$ as $x \mapsto a$ if there exist $\delta, M \in \R$ such that $\abs{f(x)} \leq Mg(x)$ when $0 < \abs{x-a} \leq \delta$.
For example, if $a=0$ (which will always be the case in here), we have that if $\alpha \geq \beta$, then $x^{\alpha} = \mathcal{O}(x^{\beta})$.

It will sometimes be useful to write the Fourier expansions only up to some coefficient.
For the $q$-series up to $m$, we will write $\mathcal{O}(q^m)$.
That is, if
$$
f(q) = \sum_{n \in \N} c(n) q^n,
$$
then
$$
f(q) = \sum_{n = 0}^{m-1} c(n) q^n \ + \mathcal{O}(q^m).
$$


\subsubsection{Typical Modular Forms Fourier Expansion}
\paragraph{Fourier Expansions of $G_k$}
The modular forms $G_k$ have the following $q$-expansion \cite[p.92]{CourseInArithmetic}:
$$
G_k(q) = 2\zeta(2k) + 2 \frac{{(2 \pi i)}^{2k}}{(2k-1)!} \sum_{n=1}^{\infty} \sigma_{2k-1}(n)q^n,
$$
where $\sigma_d$ is the generalized divisor function defined by:
$$
\sigma_d(n) = \sum_{m \mid n} m^d.
$$

\paragraph{Fourier Expansion of $\Delta$}
We also have \cite[p.95]{CourseInArithmetic}:
$$
\Delta(q) = q \prod_{n=1}^{\infty} (1-q^n)^{24}.
$$



\subsection{A Basis for Modular Forms}
\label{BasisModularForms}
We established that $M_k$ form a vector space over the complex numbers. 
One may ask then a basis for this vector space.

We would like to find a basis for each vector space $M_k$. It turns out that the modular forms $G_2$ and $G_3$ introduced before in fact generate a basis for all $M_k$. 
It is not obvious and may in fact seem wrong at a first glance: $G_2$ and $G_3$ are modular forms of weight $4$ and $6$, whereas $M_k$ in general have modular forms of weight $2k$.
However, by taking combinations of $G_2$ and $G_3$, we may obtain modular forms of any weight $2k$. It is important to remember that when multiplied, the weight of modular forms add up.

\begin{theorem}\cite[Theorem 2.17]{ModularFormsComputationalApproach}.
    The set $S = \{G_2^aG_3^b \mid a,b \in \N \footnote{In this paper, $0 \in \N$, i.e. $\N = \{ 0, 1, 2, 3, 4, \dots \}$.}, 2a+3b = k\}$ is a basis for $M_k$. 
\end{theorem}
\begin{proof}
    Of course, the cases when $\dim(M_k)=0$ (for $k<0$ and $k=1$) are trivial, as the basis is empty, and $2a+3b = k$ has no solution for $a,b \in \N$.
    To show $S$ is a basis, we need it to span $M_k$ and to be linearly independent.
    
    We start with spanning, and we proceed by induction on $k$, with step $6$.
    As $\dim(M_k)=1$ for $k=0,2,3,4,5,7$, and the equation $2a+3b = k$ has exactly one solution for $a,b \in \N$ (namely $(a,b)=(0,0), (1,0), (0,1), (2,0), (1,1), (2,1)$), $S$ has only one element, which must be the basis.
    Now, for $k>7$, take some $a,b \in \N$ such that $2a+3b=k$. Let $f \in M_k$, and $g = G_2^aG_3^b \in M_k$.
    We have$g(\infty) \neq 0$ as none of $G_2$ or $G_3$ is a cusp form. 
    So there must be a complex $\lambda$ such that $f - \lambda g$ is a cusp form. 
    Then $f - \lambda g \in M_k \cong M_{k-6}^0$ and we can find a $h \in M_{k-6}^*$ such that $h \cdot \Delta = f - \lambda g$.
    
    By induction, $h$ must be a polynomial of $G_2$ and $G_3$; by definition, $\Delta$ is one as well (note that yet, we don't put any restriction on powers of $G_2$ and $G_3$, other then being positive integers).
    Therefore, $f = \Delta \cdot h + \lambda g$ is a polynomial of $G_2$ and $G_3$.
    From the fact that $f \in M_k$ (i.e. $f$ has weight $2k$), terms of $f$ as a polynomial of $G_2$ and $G_3$ have the from $G_2^aG_3^b$ with $2a+3b=k$.
    
    We now want to show linear independence, we proceed by contradiction.
    Suppose there is a non-trivial linear relation of terms $G_2^aG_3^b$. 
    We can multiply it by suitable $G_2$ and $G_3$ so that all terms have the form $2a+3b = k \equiv 0 \bmod 12$.
    Then, we can divide all terms by $G_3^2$, witch gives us that there is a polynomial for which $\nicefrac{G_2^3}{G_3^2}$ is a root.
    In particular, this polynomial is constant when $\nicefrac{G_2^3}{G_3^2}$ is plugged.
    This contradicts the fact that $q$-expansion of $\nicefrac{G_2^3}{G_3^2}$ is not constant.
\end{proof}

This set of makes to be a basis, and one may even find it pleasant: given the two modular forms $G_2$ and $G_3$, this set generates all the modular forms of weight $2k$ that we could think of, if we only knew these two modular forms.



\subsection{Hecke Operators}
\label{DefHeckeOperators}
We define the \textit{Hecke operators} for a modular form $f$ as follows \cite[p.100]{CourseInArithmetic}:
$$
T_nf(z) = n^{2k-1}\sum_{a \geq 1,\, ad=n,\, 0 \leq b < d} d^{-2k}f \left( \frac{az+b}{d} \right)
$$
with $n \in \N$.
We can check that $T_nf$ is modular if $f$ is (as the sum of modular forms).
We may as well write $T_nf$ as a $q$-expansion as follows \cite[p.100]{CourseInArithmetic}: if $f(z) = \sum_{n \in \Z} c(n)q^n$, then
$$
T_nf(z) = \sum_{m \in \Z} \gamma(m)q^m
\quad \text{ with } \quad 
\gamma(z) = \sum_{a | (n,m),\, a \geq 1} a^{2k-1} c\left( \frac{mn}{a^2} \right).
$$
