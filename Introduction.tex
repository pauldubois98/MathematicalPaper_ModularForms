% !TeX spellcheck = en_GB
\setcounter{section}{-1}
\section{Introduction}

Our main goal will be to find new governing fields for the maps $a_{ij}$ involved in the expression of Hecke operators as power series of $T_3$ and $T_5$.
This will lead us to a new conjecture.



In section 1, we will introduce modular forms as functions of the upper half-plane.
We will present standard examples and basic properties of these modular forms.
Then we will define Hecke operators $T_n$ on modular forms.

Section 2 is dedicated to the reduction of modular forms modulo 2.
We will build up the theory and effectively perform this reduction.
We will do so first on modular forms, and then on the Hecke operators.
Our interest will be limited to Hecke operators for prime numbers ($T_p$).
It is important to determine if the reduction of Hecke operators modulo 2 becomes trivial.
In fact, it is never trivial.
We will prove this via the classical way, and as a consequence of a property that will be new to this paper.
As the reduction is not trivial, this yields an interesting theory.
We will, therefore, study the properties of Hecke operators modulo two.
The main one is that, given a modular form, Hecke operators are all nilpotent.
Finding the order of nilpotency of a modular form is a solved problem, that we will go through.

Section 3 is then dedicated to the Hecke algebra (the space of Hecke operators) modulo 2.
They are generated by power series of $T_3$ and $T_5$, and we will prove it.
The coefficients of this power series for $T_p$ are denoted $a_{ij}(p)$.
They will play a major role in the rest of the paper.

In section 4, we introduce all the algebraic number theory that will be needed for the rest of the paper.
First, we define the basis of algebraic number theory, that is field extensions, ideals, prime ideals, splitting, ramification and inertia.
Then we will look at Frobenius elements, together with their basic properties.
Finally, we mention Chebotarev's density theorem.

Then we will make the link between the last two sections (3 and 4): the coefficients $a_{ij}(p)$, seen as a map, are Frobenian.
The first cases (for small $i+j \leq 2$) have already been studied, and we know what the corresponding governing fields are.
Our goal is to investigate the possible governing fields for other cases (that is, for $i+j>2$).
We will analyse extensions numerically.
After this numerical analysis, we will come up with very strong candidates as governing fields for $a_{0k}(p)$ and $a_{k0}(p)$ for $k \leq 7$.
The potential governing fields are extensions of each other.
When drawing a diagram as in \ref{diagramFieldsExtensions}, they appear on the \textit{extreme sides}.
It appears that the Galois group of all diagonal governing fields that are found are dihedral groups.
The fact that this is verified not only for $k \leq 2$, but in fact $k \leq 7$ leads us to a new conjecture: the on diagonal governing groups \ref{diagonalGoverningGroupsConjecture}.

In the last section, we will discuss computational methods.
The results gave enough data to analyse the fields extensions, which was the goal of the paper.
To get this data, various techniques will be involved.
We will use the structure of the mathematical objects, together with careful programming to create a new computing technique (so-called "exact computations").
It allows, in the case of modular forms modulo 2, to recover the numerical error made when calculating Hecke operator of forms.
We will combine this with a high-performance language (Julia), and a high-performance library to compute fields extensions (PARI GP).

In the appendix, a subset of the tables from the computations made can be found.
The most important parts of the code used are also in the appendix.
Tables and code have their full versions available online (the corresponding links are in the appendix as well).
The governing fields calculated may be found at the very end of the appendix.
