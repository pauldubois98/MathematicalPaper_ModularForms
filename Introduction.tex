% !TeX spellcheck = en_GB
\setcounter{section}{-1}
\section{Introduction}

Following \cite{OrdreNilpotenceOperateurHecke}, we define a \textit{modular form modulo} $2$ \textit{of level} $1$ to be a polynomial in the power series 
$$
\Delta(q) = \sum_{n = 0}^{\infty}q^{(2n+1)^2} \in \F_2\left[\left[ q \right]\right].
$$
After we introduce the classical theory of modular forms as functions on the upper half-plane in Section 1, we will provide justification for this etymology by constructing certain suitable integral bases for the spaces of modular forms of level $1$.

We will then be interested in the reduction of Hecke operators modulo $2$.
Our interest will be limited to Hecke operators for odd prime numbers ($T_p$).
We will prove that this reduction is never trivial as a consequence of a new property that will be proven in this paper.
The main property of Hecke operators on modular forms modulo $2$ is that, given a modular form, they are all nilpotent.
Finding the order of nilpotency of a modular form is a solved problem, that we will go through.

The main object of our study is the $\F_2$-algebra $A$ generated by Hecke operators $T_p$, $p$ odd, acting on the $\F_2$-space spanned by odd powers of $\Delta$.
The Hecke algebra $A$ is isomorphic to the power series ring 
$$
\F_2[[T_3, T_5]].
$$
A key feature of this presentation of $A$ is that the coefficients of the power series corresponding to $T_p$ are Frobenian functions of $p$.
More precisely, for each odd prime $p$, let us write 
$$
T_p = \sum{i+j \geq 1} a_{ij}(p)T_3^iT_5^j
$$
Then, as a consequence of a theorem of \cite{bellaiche}, for each fixed $i$ and $j$, there is a normal extension $M_{ij}/\Q$ such that the Frobenius conjugacy class of $p$ in $M_{ij}/\Q$ determines the value $a_{ij}(p) \in \F_2$.
For instance, in \cite{OrdreNilpotenceOperateurHecke}, they claim that one can take 
$$
M_{01} = \Q\left( \zeta_8 \right)
\quad \text{ and } \quad 
M_{02} = \Q\left( \zeta_8, \sqrt[4]{2} \right)
$$
We carry out with various high-performance computations techniques.
The structure of modular forms modulo 2 will in fact lead to a new computing technique (so-called "exact computations").
This allows us to find new results, for instance, we have computed that
\begin{multline*}
	M_{03} = \mathbb{Q}\left(\mu, \sqrt[4]{2}, \sqrt{
		- \frac{3136435454775881 \sqrt[4]{2}}{562949953421312} 
		+ \frac{4208721080340285 \sqrt{2}}{2251799813685248} 
	}\right. \\
	\left. \overline{ 
		+ \frac{3672578267558083 \cdot \sqrt[4]{2}^3}{562949953421312} 
		+ \frac{3582104167901087}{281474976710656}
	}\right).
\end{multline*}
A complete list of the governing fields calculated may be found at the very end of the appendix \ref{governingFieldsResults} or \href{https://pauldubois98.github.io/HeckeOperatorsModuloTwo/GoverningFields/}{online} at \url{https://pauldubois98.github.io/HeckeOperatorsModuloTwo/GoverningFields/}.

This will leads us to the following new conjecture:
\ref{diagonalGoverningGroupsConjecture}
\begin{conjecture}[Diagonal Governing Groups Conjecture]
	For all $k \in \N^*$, there exists a field $M_{0k}$ such that $M_{0k}$ is a governing field for $a_{0k}$, and $G_{0k} = \Gal{M_{0k}/\Q}$ is dihedral.
	For all $k \in \N^*$, there exists a field $M_{k0}$ such that $M_{k0}$ is a governing field for $a_{k0}$, and $G_{k0}$ is dihedral.
	Moreover $M_{k0} \neq M_{0k}$ in general, but $G_{k0} = G_{0k}$.
\end{conjecture}
