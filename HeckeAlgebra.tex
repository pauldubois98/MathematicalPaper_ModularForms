% !TeX spellcheck = en_GB
\section{Hecke Algebra}
This section follows from \cite{StructureAlgebreHecke}.

We recall $\mathcal{F} = \left\langle \Delta^k \ | \ k \text{ odd} \right\rangle$, i.e. $\mathcal{F} = \left\langle \Delta, \Delta^3, \Delta^5, \Delta^7, \Delta^9, \dots \right\rangle$
\subsection{Definition}
We redefine $ \mathcal{F}(n) = \left\langle \Delta, \Delta^3, \Delta^5, \dots, \Delta^{2n-1} \right\rangle $ so that $\dim(\mathcal{F}(n)) = n$

We define $A(n)$ as the $\F_2$-subalgebra of $\End{\mathcal{F}(n)}$ given by $\F_2$ and $T_p$. That is, if 
$
\mathfrak{m}(n) = \{T_{p_1} \cdot T_{p_2} \cdots T_{p_k} | p_1, p_2, \dots, p_k \in \primes, k\geq 1\}
$
is a sub-vector-space of $\mathcal{F}$, we get $A(n) = \F_2 \oplus \mathfrak{m}(n)$.

\begin{property}
	$\mathfrak{m}(n)$ is the only maximum ideal of $A(n)$.
\end{property}
\begin{proof}
	Firstly, we note that $\nicefrac{A(n)}{\mathfrak{m}} \cong \F_2$.
	Since $\F_2$ is a field, $\mathfrak{m}$ must be a maximum ideal.
	
	Now, suppose $I$ is an other (i.e. $I \neq \mathfrak{m}$) maximum ideal of $A(n)$.
	Then there is an operator $u \in \mathfrak{m}$ such that $(1+u) \in I$.
	Since Hecke operators are nilpotent \ref{NilpotencyHeckeOperators}, there exists $n \in \N$ such that $u^n=0$.
	By induction, $(1+u^n) \in I$ for all $n \geq 1$.
	% Suppose $(1+u^n) \in I$. As $(1+u) \in I$ we then get $(1+u)(1+u^n)+(1+u)+(1+u^n) = 1+u+u^n+u^{n+1}+1+u+1+u^n = 1+u^{n+1}] \in I$
\end{proof}

Note that as Hecke operators are all nilpotent \ref{NilpotencyHeckeOperators}, the ideal $\mathfrak{m}(n)$ is itself nilpotent.
In fact, from the minimum \ref{MinimumOrderNilpotencyHeckeOperators} and maximum \ref{MaximumOrderNilpotencyHeckeOperators} nil-potency order property extend to the ideal $\mathfrak{m}$.

Let the dual of $\mathbb{F}(n)$ be $\mathcal{F}(n)^* = \{ F: \mathcal{F} \to \F_2 \}$.
Then $\mathbb{F}(n)^*$ is an $A(n)$-module with operation 
$ (u \cdot F)(f) = F(u | f) $ for $u \in A(n)$ and $F \in \mathcal{F}(n)$.

We define $e_n$ to be the element of $\mathcal{F}(n)$ such that $e_n(\Delta) = 1$ and $e_n(\Delta^{2j+1}) = 0$ for all $1 \leq j <n$ (i.e. characteristic of $\Delta$).\\
Denote the $q$-coefficients of a modular form $f$ by $a_m(f)$, so $f = \sum_{m>0} a_m(f)q^m$.
Then $e_n(f) = a_1(f)$.
For an odd prime $p$, have $a_1(T_p|f) = a_p(f)$, so $T_p \cdot e_n(f) = a_p(f)$.
By induction, this gives for odd primes $p_1, p_2, \dots, p_k$:
$$
T_{p_1} T_{p_2} \cdots T_{p_k} \cdot e_n(f) 
= a_{p_1 p_2\cdots p_k}(f)
$$

\subsection{Basic Properties}

\begin{property}
	Note that for a non zero modular forms $f \in \mathcal{F}(n)$, there exists an operator $u \in A(n)$ such that $e_n(u|f) = 1$.
\end{property}
\begin{proof}
	We can write $f = q^m + \mathcal{O}(q^{m+1})$ for some $m$ odd (as $\mathcal{F}(n)$ is generated by odd powers of $\Delta$).
	Now, as $m$ is odd, $m=p_1 p_2 \cdots p_k$ with $p_i$ odd primes for all $1 \leq i \leq k$.
	Then, by the above, $T_{p_1} T_{p_2} \cdots T_{p_k} \cdot e_n(f) = a_m(f) = 1$.
	Letting $u = T_{p_1} T_{p_2} \cdots T_{p_k}$, we have $e_n(u|f) = u \cdot e_n(f) = 1$.
\end{proof}

\begin{property}
	$\mathcal{F}(n)^*$ is free as an $A(n)$-module, with basis $e_n$;	
	i.e., $\mathcal{F}(n)^* = A(n) \cdot e_n$.
\end{property}
\begin{proof}
	By contradiction, suppose $(e_n)$ (the $A(n)$-module generated by $e_n$) isn't $\mathcal{F}(n)^*$.
	The there must be a non-zero modular form $f \in \mathcal{F}(n)$ such that $u \cdot e_n(f)=0$ for all $u \in A(n)$.
	This would contradict the last property.
	Therefore, we have $\mathcal{F}(n)^* = A(n) \cdot e_n$.
\end{proof}

\begin{corollary}
	From last property, we deduce:
	\begin{itemize}
		\item The map $\phi: A(n) \to \mathcal{F}(n)^*$ such that $\phi(u) = u \cdot e_n$ is a bijection.
		\item The dimension of $A(n)$ is $n$.
		\item There is a bijection $\phi: A(n) \to \mathcal{F}(n)^*$.
	\end{itemize}
\end{corollary}
\begin{proof}
	We prove separately:
	\begin{itemize}
		\item This follows directly from the fact that $\mathcal{F}(n)^* = A(n) \cdot e_n$.
		\item We have $\dim(A(n)) = \dim(\mathcal{F}(n)^*)$ by the above, and $\dim(\mathcal{F}(n)^*) = \dim(\mathcal{F}(n))$ by duality.
		\item Since $\mathcal{F}(n)^* \leftrightarrow A(n)$
		\footnote{$A \leftrightarrow B$ means there exists a bijection from $A$ to $B$}, $\mathcal{F}(n)^{**} \leftrightarrow A(n)^*$.
		And as $\mathcal{F}(n)^{**} \cong \mathcal{F}(n)$, we have $\mathcal{F}(n) \leftrightarrow A(n)^*$.
	\end{itemize}
\end{proof}


\subsection{Generated by $T_3$ and $T_5$}






