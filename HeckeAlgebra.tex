% !TeX spellcheck = en_GB
\section{Hecke Algebra}
This section follows from \cite{StructureAlgebreHecke}.

We recall $\mathcal{F} = \left\langle \Delta^k \ | \ k \text{ odd} \right\rangle$, i.e. $\mathcal{F} = \left\langle \Delta, \Delta^3, \Delta^5, \Delta^7, \Delta^9, \dots \right\rangle$
\subsection{Definition}
We redefine $ \mathcal{F}(n) = \left\langle \Delta, \Delta^3, \Delta^5, \dots, \Delta^{2n-1} \right\rangle $ so that $\dim(\mathcal{F}(n)) = n$

We define $A(n)$ as the $\F_2$-subalgebra of $\End{\mathcal{F}(n)}$ given by $\F_2$ and $T_p$. That is, if 
$
\mathfrak{m}(n) = \{T_{p_1} \cdot T_{p_2} \cdots T_{p_k} | p_1, p_2, \dots, p_k \in \primes, k\geq 1\}
$
is a sub-vector-space of $\mathcal{F}$, we get $A(n) = \F_2 \oplus \mathfrak{m}(n)$.

\begin{property}
	$\mathfrak{m}(n)$ is the only maximum ideal of $A(n)$.
\end{property}
\label{AisLocalAlgebra}
\begin{proof}
	Firstly, we note that $\nicefrac{A(n)}{\mathfrak{m}} \cong \F_2$.
	Since $\F_2$ is a field, $\mathfrak{m}$ must be a maximum ideal.
	
	Now, suppose $I$ is an other (i.e. $I \neq \mathfrak{m}$) maximum ideal of $A(n)$.
	Then there is an operator $u \in \mathfrak{m}$ such that $(1+u) \in I$.
	Since Hecke operators are nilpotent \ref{NilpotencyHeckeOperators}, there exists $n \in \N$ such that $u^n=0$.
	By induction, $(1+u^n) \in I$ for all $n \geq 1$.
	% Suppose $(1+u^n) \in I$. As $(1+u) \in I$ we then get $(1+u)(1+u^n)+(1+u)+(1+u^n) = 1+u+u^n+u^{n+1}+1+u+1+u^n = 1+u^{n+1}] \in I$
\end{proof}

Note that as Hecke operators are all nilpotent \ref{NilpotencyHeckeOperators}, the ideal $\mathfrak{m}(n)$ is itself nilpotent.
In fact, from the minimum \ref{MinimumOrderNilpotencyHeckeOperators} and maximum \ref{MaximumOrderNilpotencyHeckeOperators} nil-potency order property extend to the ideal $\mathfrak{m}$.

Let the dual of $\mathbb{F}(n)$ be $\mathcal{F}(n)^* = \{ F: \mathcal{F}(n) \to \F_2 \}$.
Then $\mathbb{F}(n)^*$ is an $A(n)$-module with operation 
$ (u \cdot F)(f) = F(u | f) $ for $u \in A(n)$ and $F \in \mathcal{F}(n)$.

We define $e_n$ to be the element of $\mathcal{F}(n)$ such that $e_n(\Delta) = 1$ and $e_n(\Delta^{2j+1}) = 0$ for all $1 \leq j <n$ (i.e. characteristic of $\Delta$).\\
Denote the $q$-coefficients of a modular form $f$ by $a_m(f)$, so $f = \sum_{m>0} a_m(f)q^m$.
Then $e_n(f) = a_1(f)$.
For an odd prime $p$, have $a_1(T_p|f) = a_p(f)$, so $T_p \cdot e_n(f) = a_p(f)$.
By induction, this gives for odd primes $p_1, p_2, \dots, p_k$:
$$
T_{p_1} T_{p_2} \cdots T_{p_k} \cdot e_n(f) 
= a_{p_1 p_2\cdots p_k}(f)
$$

To fit naturally with the above definitions, we define the Hecke algebra $A$ as follows:
As $\mathcal{F}(n) \subset \mathcal{F}(n+1)$, we can restrict elements of $A(n+1)$ to $\mathcal{F}$ to obtain an element of $A(n)$.
If we consider the map $\phi_n: A(n+1) \to A(n)$ to be the restriction to $\mathcal{F}(n)$, then $\phi_n$ is a homomorphism.
As $A(1)$ is either identity or zero, $A(1) \cong \F_2$.
Therefore, we have the chain:
$$
\dots \to A(n+1) \to A(n) \to A(n-1) \to \dots \to A(2) \to A(1) \cong \F_2
$$
We then define the Hecke algebra $A$ to be the projective limit of the above $A(n)$ as $n \to \infty$.
Explicitly, this means
$$
A = \varprojlim_{n \in \N} A(n) = \left\lbrace T_{p_1} \cdot T_{p_2} \cdots T_{p_k} | p_1, p_2, \dots, p_k \in \primes, k\geq 0 \right\rbrace .
$$.



\subsection{Basic Properties}

\begin{property}
	Note that for a non zero modular forms $f \in \mathcal{F}(n)$, there exists an operator $u \in A(n)$ such that $e_n(u|f) = 1$.
\end{property}
\begin{proof}
	We can write $f = q^m + \mathcal{O}(q^{m+1})$ for some $m$ odd (as $\mathcal{F}(n)$ is generated by odd powers of $\Delta$).
	Now, as $m$ is odd, $m=p_1 p_2 \cdots p_k$ with $p_i$ odd primes for all $1 \leq i \leq k$.
	Then, by the above, $T_{p_1} T_{p_2} \cdots T_{p_k} \cdot e_n(f) = a_m(f) = 1$.
	Letting $u = T_{p_1} T_{p_2} \cdots T_{p_k}$, we have $e_n(u|f) = u \cdot e_n(f) = 1$.
\end{proof}

\begin{property}
	$\mathcal{F}(n)^*$ \footnote{$\mathcal{F}(n)^*$ denotes the dual of $\mathcal{F}(n)$.} is free as an $A(n)$-module, with basis $e_n$;	
	i.e., $\mathcal{F}(n)^* = A(n) \cdot e_n$.
\end{property}
\begin{proof}
	By contradiction, suppose $(e_n)$ (the $A(n)$-module generated by $e_n$) isn't $\mathcal{F}(n)^*$.
	The there must be a non-zero modular form $f \in \mathcal{F}(n)$ such that $u \cdot e_n(f)=0$ for all $u \in A(n)$.
	This would contradict the last property.
	Therefore, we have $\mathcal{F}(n)^* = A(n) \cdot e_n$.
\end{proof}

\begin{corollary}
	From last property, we deduce:
	\begin{itemize}
		\item The map $\phi: A(n) \to \mathcal{F}(n)^*$ such that $\phi(u) = u \cdot e_n$ is a bijection.
		\item The dimension of $A(n)$ is $n$.
		\item There is a bijection $\phi: A(n) \to \mathcal{F}(n)^*$.
	\end{itemize}
\end{corollary}
\begin{proof}
	We prove separately:
	\begin{itemize}
		\item This follows directly from the fact that $\mathcal{F}(n)^* = A(n) \cdot e_n$.
		\item We have $\dim(A(n)) = \dim(\mathcal{F}(n)^*)$ by the above, and $\dim(\mathcal{F}(n)^*) = \dim(\mathcal{F}(n))$ by duality.
		\item Since $\mathcal{F}(n)^* \leftrightarrow A(n)$
		\footnote{$A \leftrightarrow B$ means there exists a bijection from $A$ to $B$}, $\mathcal{F}(n)^{**} \leftrightarrow A(n)^*$.
		And as $\mathcal{F}(n)^{**} \cong \mathcal{F}(n)$, we have $\mathcal{F}(n) \leftrightarrow A(n)^*$.
	\end{itemize}
\end{proof}



\subsection{As generated by $T_3$ and $T_5$}
The Hecke algebra is in fact generated by powers of $T_3$ and $T_5$, i.e., we have:
$$
A = \F_2 \left[\left[ T_3, T_5 \right]\right] 
$$
The strategy to show this splits in two parts:
First, we show that $A(n) = \F_2 \left[ T_3, T_5 \right]$ (with $T_3$ and $T_5$ seen in $A(n)$).
Then, we show that this equation remains when taking the limit.

\begin{property}
	We have $A(n) = \F_2 \left[ T_3, T_5 \right]$.
\end{property}
\begin{proof}
	We define $A'(n) = \F_2 \left[ T_3, T_5 \right]$ as a subalgebra of $A(n)$.
	This is a local algebra (i.e. it has a unique maximum ideal).
	The idea here is similar to the one involved in $A$ being a local algebra (see \ref{AisLocalAlgebra}).
	We denote the maximal ideal of $A'(n)$ by $\mathfrak{m}'$.
	
	Suppose, for contradiction, that $A' \neq A$, so $\dim(A') < n$.
	
	If $\mathcal{F}(n)^*$ (seen as a $A'(n)$-module), was cyclic (i.e. generated by a single element), then $\dim(\mathcal{F}(n)^*) < n$.
	However, we know $\dim(\mathcal{F}(n)^*)=n$, so $\mathcal{F}(n)^*$ isn't cyclic.
	% cyclic / monogenous means the module is generated by a single element
	
	We define $V=\nicefrac{\mathcal{F}(n)^*}{\mathfrak{m}'\mathcal{F}(n)^*}$.
	Nakayama lemma under the statement for maximal ideals implies that $V$ has dimension $>1$.
	% using https://en.wikipedia.org/wiki/Nakayama%27s_lemma#Local_rings
	
	
	Now we put $U = \{ f\in \mathcal{F}(n) \ | \ \ a|f=0 \quad \forall a \in \mathfrak{m}' \}$, i.e. the modular forms that are zero after application of any operator in $\mathfrak{m}'$.
	We want to show that this vector space $U$ has dimension $>1$.
	
	As vector spaces, we know: $\left( \nicefrac{W_1}{W_2} \right)^* \cong W_2^0$, so:
	$$
	\left( \frac{\mathcal{F}(n)^*}{\mathfrak{m}'\mathcal{F}(n)^*} \right)^* \cong \left( \mathfrak{m}'\mathcal{F}(n)^* \right)^0
	$$
	Where 
	$$
	\left( \mathfrak{m}'\mathcal{F}(n)^* \right)^0 = \{ \tilde{f} \in \mathcal{F}(n)^{**} \ | \ \tilde{f}(a)=0 \quad \forall a \in \mathfrak{m}'\mathcal{F}(n)^* \}
	$$
	is the inhalator of $\mathfrak{m}'\mathcal{F}(n)^*$.
	We know there is an isomorphism between $\mathcal{F}(n)^{**}$ and $\mathcal{F}(n)$\footnote{Just take $\phi: \mathcal{F}(n) \to \mathcal{F}(n)^{**}$ such that $f \to \left[ F \to F(f) \right]$.}.
	Thus, we have:
	$$
	\left( \mathfrak{m}'\mathcal{F}(n)^* \right)^0 \cong \{ f \in \mathcal{F}(n) \ | \ \ a|f=0 \quad \forall a \in \mathfrak{m}'\mathcal{F}(n)^* \}
	$$
	But one may recognize that this is exactly $U$.
	
	Therefore, 
	$\dim(U) 
	= \dim(\left( \mathfrak{m}'\mathcal{F}(n)^* \right)^0)
	= \dim(V) = \dim(V^*) > 1
	$
	
	$\{ 0, \Delta \}$ is a subspace of $U$ with dimension 1.
	As $\dim(U)>1$, there must exists a modular form $f \in \mathcal{F}(n)$ such that:
	$f$ is neither $0$ or $\Delta$, and $a|f=0$ for all $a \in \mathfrak{m}'$.
	In particular, this means that $T_3|f=0$ and $T_5|f=0$.
	However, it contradicts 5.3 of \cite{OrdreNilpotenceOperateurHecke}.
	
	Thus we have $A(n) = A'(n) = \F_2 \left[ T_3, T_5 \right]$.
\end{proof}

\begin{property}
	We have $A = \F_2 \left[\left[ T_3, T_5 \right]\right]$.
\end{property}
\begin{proof}
	For any $n$, there is a homomorphism $\psi_n: \F_2 \left[ x, y \right] \to A(n)$ with $\psi_n(x) = T_3$ and $\psi_n(x) = T_5$.
	Therefore, we can take the limit as $n \to \infty$ to get a homomorphism $\psi: \F_2 \left[\left[ x, y \right]\right] \to A$ such that $\psi(x) = T_3$ and $\psi(x) = T_5$.
	
	The fact that $\psi$ is surjective follows directly from the last property.
	
	Now we want to show that $\psi$ is injective.
	Since is is already homomorphic, it suffices to show that for any element $u = \sum_{i,j} \lambda_{ij}T_3^iT_5^j | f = \Delta$.
	Note that this sum id finite, as Hecke operators are nilpotent.\\
	If $\lambda_{00}=1$, take $f=\Delta$.
	Suppose $\lambda_{00}=0$: Consider $S = \{ (i,j) \in \N^2 \ | \ \lambda_{ij}=1 \}$.\\
	Let $S_{min} = \{ (m,n) \ | \ m+n \leq i+j \quad \forall (i,j) \in S \}$ and $(a,b) \in S_{min}$ be such that $b \leq n \forall (m,n) \in S_{min}$.
	Let $k = [a,b]$ as in [bla], and let $f = \Delta^k$.
	[Then, from 4.3 and 4.4 of \cite{OrdreNilpotenceOperateurHecke}, we have that $T_3^aT_5^b|f=\Delta$ and $T_3^iT_5^j|f=0$ for all $(i,j) \in S$.]
	Thus, $u|f = \Delta$.
	
	Therefore, $\psi$ is an isomorphism, which completes the proof.
\end{proof}



\subsection{Table of Powers of Hecke Operators}
\resizebox{\textwidth}{!}{%
	\begin{tabular}{|r|rrrr|}
		\hline
		\textbf{$\Delta^1$} & \textbf{$T_5^0$} & \textbf{$T_5^1$} & \textbf{$T_5^2$} & \textbf{$\dots$} \\\hline
		$T_3^0$ & $\Delta^1$ & 0 & 0 & $\dots$ \\
		$T_3^1$ & 0 & 0 & 0 & $\dots$ \\
		$T_3^2$ & 0 & 0 & 0 & $\dots$ \\
		$\vdots$ & $\vdots$ & $\vdots$ & $\vdots$ & $\ddots$ \\
		\hline
	\end{tabular}
	\begin{tabular}{|r|rrrr|}
		\hline
		\textbf{$\Delta^3$} & \textbf{$T_5^0$} & \textbf{$T_5^1$} & \textbf{$T_5^2$} & \textbf{$\dots$} \\\hline
		$T_3^0$ & $\Delta^3$ & 0 & 0 & $\dots$ \\
		$T_3^1$ & $\Delta^1$ & 0 & 0 & $\dots$ \\
		$T_3^2$ & 0 & 0 & 0 & $\dots$ \\
		$\vdots$ & $\vdots$ & $\vdots$ & $\vdots$ & $\ddots$ \\
		\hline
	\end{tabular}
	\begin{tabular}{|r|rrrr|}
		\hline
		\textbf{$\Delta^5$} & \textbf{$T_5^0$} & \textbf{$T_5^1$} & \textbf{$T_5^2$} & \textbf{$\dots$} \\\hline
		$T_3^0$ & $\Delta^5$ & $\Delta^1$ & 0 & $\dots$ \\
		$T_3^1$ & 0 & 0 & 0 & $\dots$ \\
		$T_3^2$ & 0 & 0 & 0 & $\dots$ \\
		$\vdots$ & $\vdots$ & $\vdots$ & $\vdots$ & $\ddots$ \\
		\hline
	\end{tabular}
	\begin{tabular}{|r|rrrr|}
		\hline
		\textbf{$\Delta^7$} & \textbf{$T_5^0$} & \textbf{$T_5^1$} & \textbf{$T_5^2$} & \textbf{$\dots$} \\\hline
		$T_3^0$ & $\Delta^7$ & $\Delta^3$ & 0 & $\dots$ \\
		$T_3^1$ & $\Delta^5$ & $\Delta^1$ & 0 & $\dots$ \\
		$T_3^2$ & 0 & 0 & 0 & $\dots$ \\
		$\vdots$ & $\vdots$ & $\vdots$ & $\vdots$ & $\ddots$ \\
		\hline
	\end{tabular}
}
\resizebox{\textwidth}{!}{%
	\begin{tabular}{|r|rrrrr|}
		\hline
		\textbf{$\Delta^9$} & \textbf{$T_5^0$} & \textbf{$T_5^1$} & \textbf{$T_5^2$} & \textbf{$T_5^3$} & \textbf{$\dots$} \\\hline
		$T_3^0$ & $\Delta^9$ & 0 & 0 & 0 & $\dots$ \\
		$T_3^1$ & $\Delta^3$ & 0 & 0 & 0 & $\dots$ \\
		$T_3^2$ & $\Delta^1$ & 0 & 0 & 0 & $\dots$ \\
		$T_3^3$ & 0 & 0 & 0 & 0 & $\dots$ \\
		$\vdots$ & $\vdots$ & $\vdots$ & $\vdots$ & $\vdots$ & $\ddots$ \\\hline
	\end{tabular}
	\begin{tabular}{|r|rrrrr|}
		\hline
		\textbf{$\Delta^{11}$} & \textbf{$T_5^0$} & \textbf{$T_5^1$} & \textbf{$T_5^2$} & \textbf{$T_5^3$} & \textbf{$\dots$} \\\hline
		$T_3^0$ & $\Delta^{11}$ & 0 & 0 & 0 & $\dots$ \\
		$T_3^1$ & $\Delta^9$ & 0 & 0 & 0 & $\dots$ \\
		$T_3^2$ & $\Delta^3$ & 0 & 0 & 0 & $\dots$ \\
		$T_3^3$ & $\Delta^1$ & 0 & 0 & 0 & $\dots$ \\
		$\vdots$ & $\vdots$ & $\vdots$ & $\vdots$ & $\vdots$ & $\ddots$ \\\hline
	\end{tabular}
	\begin{tabular}{|r|rrrrr|}
		\hline
		\textbf{$\Delta^{13}$} & \textbf{$T_5^0$} & \textbf{$T_5^1$} & \textbf{$T_5^2$} & \textbf{$T_5^3$} & \textbf{$\dots$} \\\hline
		$T_3^0$ & $\Delta^{13}$ & $\Delta^9$ & 0 & 0 & $\dots$ \\
		$T_3^1$ & $\Delta^7$ & $\Delta^3$ & 0 & 0 & $\dots$ \\
		$T_3^2$ & $\Delta^5$ & $\Delta^1$ & 0 & 0 & $\dots$ \\
		$T_3^3$ & 0 & 0 & 0 & 0 & $\dots$ \\
		$\vdots$ & $\vdots$ & $\vdots$ & $\vdots$ & $\vdots$ & $\ddots$ \\\hline
	\end{tabular}
}
\begin{center}
	Action of Powers Hecke Operators $T_3$ and $T_5$ on Modular Forms Modulo 2 (up to $\Delta^{13}$).
\end{center}
A larger table may be found in the appendix (see \ref{table:PowersHeckeOperators}).
