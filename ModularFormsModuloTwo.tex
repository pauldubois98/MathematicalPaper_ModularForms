\section{Modular Forms Modulo Two}
\subsection{Strategy to Reduce Modulo Two}
It is not trivial, at this point, why and how we can reduce modulo 2 modular forms, objects that have coefficients in $\C$.
In general, reduction modulo a number is only possible with whole numbers (integers).
We would like to reduce modulo 2 coefficients of the Fourier series for modular forms.
But at the moment, they lie in $\C$.

In fact, we will introduce a new basis for the modular forms: the so called Miller Basis.
The coefficients of all the forms in this basis are integers. It is then possible to consider the space of modular forms over $\Z$ instead of $\C$.
Once this is done, we will reduce all the newly integral coefficients modulo $2$.

In this section, we will denote all objects reduced modulo 2 with an $\overline{\text{over-line}}$:
\begin{itemize}
	\item The modular form $f$ once reduced will be denoted $\overline{f}$.	\item The coefficients of the $q$-expansion $c$ will reduce to $\overline{c}$
	\item The Hecke operators $T_n$ reduced will be denoted $\overline{T_n}$.
\end{itemize}

\subsection{Integral Basis}
\subsubsection{Normalisation of Typical Modular Forms}
\paragraph{Normalisation of Eisenstein series $G_k$}
We first recall the formula for $q$ extension of $G_k$ and the one for $\zeta(2k)$:
$$
G_k(q) = 2\zeta(2k) + 2 \frac{{(2 \pi i)}^{2k}}{(2k-1)!} \sum_{n=1}^{\infty} \sigma_{2k-1}(n)q^n
$$
and
$$
2\zeta(2k) = \frac{(2\pi)^{2k}}{(2k)!}B_k
$$
so overall:
$$
G_k(q) = \frac{(2\pi)^{2k}}{(2k)!}B_k + 2 \frac{{(2 \pi i)}^{2k}}{(2k-1)!} \sum_{n=1}^{\infty} \sigma_{2k-1}(n)q^n
$$

We would like to normalize this series, so that the coefficients become integers, so that we can ultimately reduce them modulo 2.
Right now, coefficients are rational.

As we want to keep the series modular with same weight, the only tool we have to normalize the series is multiplication by a constant.
The normalization is a crucial point: 
If we multiply by $2$ all coefficients of a modular form that already lie in $\Z$, the reduction $\bmod 2$ will always give zero.

First, let's normalize the series to have particular values on some coefficients of interest.
There are two justified ways to do so: normalize to have constant coefficient set to one, and to have $q$ coefficient is set to one.
We will introduce both:
Let $E_k$ be such that:
$$
E_k.2\zeta(2k) = G_k
$$
so that
$$
E_k = 1 + (-1)^k \frac{4k}{B_k} \sum_{n=1}^{\infty} \sigma_{2k-1}(n)q^n.
$$
$E_k$ then has constant coefficient set to one.

Let $F_k$ be such that:
$$
F_k.\left( 2 \frac{{(2 \pi i)}^{2k}}{(2k-1)!} \right) = G_k
$$
so that
$$
F_k =  (-1)^k \frac{B_k}{4k} + \sum_{n=1}^{\infty} \sigma_{2k-1}(n)q^n.
$$
$F_k$ then has $q$ coefficient set to one (as $\sigma_{2k-1}(1)=1$).

Clearly, the coefficients of this expansion remain in $\Q$ at least, and we will show that for some specific $k$, the coefficients lie in fact in $\Z$.
Both $F_k$ and $E_k$ are interesting, but for our purpose (reducing modulo 2), we will use $E_k$.
Note that $E_k$ are normalized versions of Eisenstein series $G_k$, but in literature, both are called Eisenstein series see \cite[p.6]{IntoductionModularFormsWorkshop} for example.

\paragraph{The Modular Discriminant $\Delta$ Normalized}
Again, we recall the formula for $q$ extension of $\Delta$:
$$
\Delta(q) = q \prod_{n=1}^{\infty} (1-q^n)^{24}
$$
Clearly, the coefficients in expansion of $\Delta$ are integers (which we can reduce modulo 2).
This is the reason why we defined $\Delta$ with the $\frac{1}{(2\pi)^{12}}$ factor in front.


\subsubsection{Miller Basis}
\paragraph{Basis with Integral Coefficients (in Fourier Series)}
Applying normalization $G_k \to E_k$ above for $k=2,3$, we get:
\begin{align*}
	E_2 &= 1 + \frac{8}{B_2} \sum_{n=1}^{\infty} \sigma_{3}(n)q^n \qquad B_2 = \frac{1}{30} \\
	    &= 1 + 240 \sum_{n=1}^{\infty} \sigma_{3}(n)q^n
\end{align*}

and
\begin{align*}
	E_3 &= 1 - \frac{12}{B_3} \sum_{n=1}^{\infty} \sigma_{5}(n)q^n \qquad B_3 = \frac{1}{42} \\
	    &= 1 - 504 \sum_{n=1}^{\infty} \sigma_{5}(n)q^n
\end{align*}

Now, we have shown that $\{G_2^aG_3^b | 2a+3b=k\}$ is a basis for modular forms of weight $2k$ over the complex (see \ref{BasisModularForms}).
As $E_2 = \lambda G_2,\ \lambda \in \C$ and $E_3 = \mu G_3,\ \mu \in \C$, we have that $\{E_2^aE_3^b | 2a+3b=k\}$ remains a basis for $M_k$ over $\C$.

It is clear, from the series, that coefficients of the $q$-expansion of both $E_2$ and $E_3$ are all integers.
Thus, so are coefficients of combinations of $E_2$ and $E_3$.
Therefore, we have found a basis for $M_k$ such that all elements in the basis have only integral coefficients in their $q$-expansion.
\label{IntegralBasisModularForms}

\paragraph{Miller Basis for $M_k^0$}
This is a nice result, but we can in fact do better, by forcing the first coefficients to chosen values.

\begin{theorem}
	For the space of modular cusp forms $M_k^0$, there exists a basis $\{f_1, \cdots, f_r\}$ such that:
	\begin{itemize}
		\item $f_i \in Z[q]$
		\item $ a_i^j = \delta_ij = 
		\left\lbrace
		\begin{array}{l l}
			1 & \text{ if } i   =  j \\
			0 & \text{ if } i \neq j
		\end{array}
		\right. \quad
		\forall 1 \leq i,j \leq r$\\
		where $a_i^j$ is the coefficient of $q^j$ in expansion of $f_i$.
	\end{itemize}
\end{theorem}
This is commonly called the Miller basis for $M_k^0$, as it was first introduced by Victor Saul Miller \cite{MillerThesis}.
\begin{proof}
	\begin{itemize}
		%boring part
		\item For $k<6$, $k=7$, we have $\dim(M_k^0)=0$.
		Thus, $\emptyset$ is a basis which satisfies the Miller basis properties.
		
		%semi-boring part
		\item For $k=6$, we have $\dim(M_k^0)=1$.
		Thus, $\{ \Delta \}$ is a basis which satisfies the Miller basis properties.
		
		%interesting part
		\item For $k \geq 7$, we let $r = \dim(M_k^0) \geq 1$.	
		We then consider the set
		$$
		\{ g_j | 1 \leq j \leq r \}
		$$
		where
		$$
		g_j = \Delta^jE_3^{2(d-j)+b}E_2^a
		$$
		with
		$$
		2a+3b \leq 7 \ \&\ 2a+3b \cong k \bmod 6$$
		$$
		\& \ \ d=\frac{k-(2a+3b)}{6} \quad \in \N \text{ as $k \geq 7$}
		$$
		Note that $a$ and $b$ are unique unless $k \cong 0 \bmod 6$. In witch case, we use by convention $a=0$, $b=0$.
		
		As all $E_2$, $E_3$, and $\Delta$ have integral coefficients, $g_j$ will as well.
		
		We then look at the $q$ series:
		$$
		\Delta(q) = q + O(k^2) \implies \Delta^j(q) = q^j + O(k^{j+1})
		$$
		As we normalized so,
		$$
		E_2(q) = 1 + O(q) \implies E_2^{\alpha}(q) = 1 + O(q)
		$$
		$$
		E_3(q) = 1 + O(q) \implies E_3^{\alpha}(q) = 1 + O(q)
		$$
		
		This gives:
		$$
		g_j(q) = q^j + O(q^{j+1}) \quad \forall 1 \leq j \leq r.
		$$
		Therefore, $\{ g_j, | 1 \leq j \leq r \}$ is clearly a linearly independent set. By dimension argument, it also spans $M_k^0$. Therefore, it forms a basis.
		Moreover, in this basis: $a_i^j = \delta_{ij} \quad i \leq j$.
		
		Finally, we can use Gaussian elimination on $\{g_j\}$ to obtain a basis $\{f_j | 1 \leq j \leq r \}$ such that: $a_i^j = \delta_{ij} \quad \forall 1 \leq i,j \leq r$.
		The coefficients will remain in $\Z$ after Gaussian elimination.
	\end{itemize}
\end{proof}

\paragraph{Extension to all $M_k$}
We already have a basis for $M_k^0$, as $\dim(M_k) = \dim(M_k^0) + 1$ (over $\C$), we just need to adjoint one element of $M_k \setminus M_k^0$ to our basis.

It was shown before that $\{E_2^aE_3^b | 2a+3b=k\}$ is a basis for $M_k$ with integral coefficients (see \ref{IntegralBasisModularForms})
One may see from the $q$-expansion that $E_2^aE_3^b = 1 + O(q)$ so $E_2^aE_3^b \in M_k \setminus M_k^0$.

Therefore, we can just add one element of $\{E_2^aE_3^b | 2a+3b=k\}$ to the Miller basis, and use Gaussian elimination again.
We get a basis foe $M_k$ of the form $\{f_j | 0 \leq j \leq r \}$ such that in this basis: $a_i^j = \delta_{ij} \quad \forall 0 \leq i,j \leq r$ (with $r=dim(M_k^0)$ i.e. $r+1=dim(M_k)$).

\paragraph{Miller Basis Examples}
\subparagraph{Miller basis for $k=16$}
We can calculate the Miller basis for $k=16$:
$k \cong 4 \bmod 12$ so $a=2$ and $b=0$; $d=2$.
We put $g_1 = \Delta^1E_3^2E_2^2$, so:
\begin{align*}
    g_1(q) &= \Delta(q)E_2^2(q)E_3^2(q)\\
           &= \left[ q - 24q^2 + 252q^3 + O(q^4) \right]\\
           & \qquad \cdot \left[ 1 + 240q + 2160q^2 + 6720q^3 + O(q^4) \right]^2\\
           & \qquad \qquad \cdot \left[ 1 - 504q - 16632q^2 + 122976q^3 + O(q^4) \right]^2\\
           &= q - 552q^2 - 188244q^3 + O(q^4)
% (Wolfram|Alpha) develop (q - 24q^2 + 252q^3)(1 + 240q + 2160q^2 + 6720q^3)^2(1 - 504q - 16632q^2 + 122976q^3)^2
\end{align*}
and $g_2 = \Delta^2E_3^0E_2^2$, so:
\begin{align*}
    g_2(q) &= \Delta^2(q)E_2^2(q)\\
           &= \left[ q - 24q^2 + 252q^3 + O(q^4) \right]^2\\
           & \qquad \cdot \left[ 1 + 240q + 2160q^2 + 6720q^3 + O(q^4) \right]^2\\
           &= q^2 + 432q^3 + O(q^4)
% (Wolfram|Alpha) develop (q - 24q^2 + 252q^3)^2(1 + 240q + 2160q^2 + 6720q^3)^2
\end{align*}
Then, $f_2=g_2$ and $f_1=g_1+552g_2$, so:
\begin{align*}
    f_1(q) &= q - 552q^2 - 188244q^3 + O(q^4) \ + \ 552 \cdot \left[q^2 + 432q^3 + O(q^4)\right] \\
           &= q + 50220q^3 + O(q^4)\\
    f_2(q) &= q^2 + 432q^3 + O(q^4)\\
% (Wolfram|Alpha) develop q - 552q^2 - 188244q^3 + 552 \cdot [q^2 + 432q^3]
\end{align*}

Therefore, up to $O(q^4)$, $\{f_1, f_2\}=\{q + 50220q^3 + O(q^4), q^2 + 432q^3 + O(q^4)\}$ is a basis for $M_{16}^0$.

To extend this base to $M_k$, we adjoint a term of the form $g_0 = E_2^aE_3^b$ where $2a+3b=16$. We pick $g_0 = E_2^8$, so:
\begin{align*}
    g_0(q) &= E_2^8(q)\\
           &= \left[ 1 + 240q + 2160q^2 + 6720q^3 + O(q^4) \right]^8\\
           &= 1 + 1920q + 1630080q^2 + 803228160q^3 + O(q^4)
% (Wolfram|Alpha) develop (1 + 240q + 2160q^2 + 6720q^3)^8
\end{align*}
Then, $f_0 = g_0 - 1920g_1 - 1630080g_2$, so:
\begin{align*}
    f_0(q) &= g_0(q) - 1920g_1(q) - 1630080g_2(q)\\
           &= \left[ 1 + 1920q + 1630080q^2 + 803228160q^3 + O(q^4) \right]\\
           & \qquad - 1920 \left[ q + 50220q^3 + O(q^4) \right] \\
           & \qquad \qquad - 1630080 \left[ q^2 + 432q^3 + O(q^4) \right]\\
           &= 1 + 2611200q^3 + O(q^4)
% (Wolfram|Alpha) develop [1 + 1920q + 1630080q^2 + 803228160q^3] - 1920\cdot[q + 50220q^3] - 1630080\cdot[q^2 + 432q^3]
\end{align*}

Therefore, up to $O(q^4)$, $\{f_0, f_1, f_2\} = \{1 + 2611200q^3 + O(q^4), q + 50220q^3 + O(q^4), q^2 + 432q^3 + O(q^4)\}$ is a basis for $M_{16}$.

\subparagraph{Miller basis for $k=92$}
The calculation of this basis may be interesting by hand once;
However, it is possible to automate it.
The procedure that calculates such coefficients is a standard in SageMath \cite{sage}.
Here is, up to $O(q^{10})$, the Miller basis for $M_{92}$:
% (sage) victor_miller_basis(92, prec=10, cusp_only=False, var='q')
\begin{align*}
	f_0 &= 1 + 3034192667130000 q^8 + 137290127714549760000 q^9 + O(q^{10})\\
	f_1 &= q + 91578443563200 q^8 + 2651503140376278561 q^9 + O(q^{10})\\
	f_2 &= q^2 + 2380310529376 q^8 + 42238207588515840 q^9 + O(q^{10})\\
	f_3 &= q^3 + 51682260816 q^8 + 530253459731160 q^9 + O(q^{10})\\
	f_4 &= q^4 + 896013480 q^8 + 4882999541760 q^9 + O(q^{10})\\
	f_5 &= q^5 + 11516000 q^8 + 28971735750 q^9 + O(q^{10})\\
	f_6 &= q^6 + 94680 q^8 + 80990208 q^9 + O(q^{10})\\
	f_7 &= q^7 + 312 q^8 - 4860 q^9 + O(q^{10})\\
\end{align*}



\subsection{Basis Modulo Two}
\subsubsection{Reduced Modular Forms}
Now that we have a basis with integral coefficients, it makes sense to reduce forms modulo 2.
For a modular form $f$, we denote its reduced modulo 2 from $\overline{f}$. It is defined as follows:

If
$$
f(q) = \sum_{n \in \N} c(n)q^n
$$
then
$$
\overline{f}(q) = \sum_{n \in \N} \overline{c}(n)q^n 
\qquad \text{ with } \ \overline{c}(n) = c(n) \bmod 2 .
$$

We want to reduce Miller basis modulo 2.
The reason is that as we know that some coefficients are ones, the reduction will not be trivial.
We will reduce separately $E_2$, $E_3$ and $\Delta$ (witch together generate the Miller basis).

\paragraph{$E_2$ reduced}
We have:
$$
E_2(q) 
= 1 + 240 \sum_{n=1}^{\infty} \sigma_{3}(n)q^n
\equiv 1 \bmod 2
$$
Therefore, the reduction modulo 2 of $E_2$ is just $1$.
We write $\overline{E_2} = 1$, so $\overline{E_2^a} = 1$, for all $a \geq 0$.

\paragraph{$E_3$ reduced}
We have:
$$
E_3(q)
= 1 - 504 \sum_{n=1}^{\infty} \sigma_{5}(n)q^n
\equiv 1 \bmod 2
$$
Therefore, the reduction modulo 2 of $E_3$ is $1$ as well.
We write $\overline{E_3} = 1$, so $\overline{E_3^b} = 1$, for all $b \geq 0$.

\paragraph{$\Delta$ reduced}
We defined before $\Delta$, and we would now like to know its $q$ extension in the standard way.
That is, an infinite sum of $q^n$, instead of an infinite product as we have at the moment.

We define the coefficients $\tau(n)$ to match in the equation: 
$$
\Delta(q) 
= q \prod_{n=1}^{\infty} (1-q^n)^{24} 
= \sum_{n=1}^{\infty} \tau(n)q^n
$$
When this holds, $\tau$ is called the Ramanujan function.

We would like an explicit formula for $\tau(n)$. More precisely, we are interested in a formula for $\tau(n) \bmod 2$.

We will calculate separately the coefficients $\tau(n) \bmod 2$ for $n$ even and odd.

\subparagraph{Case $n$ odd}
Remember $\sigma_s(n)$ as the sum of $s^{th}$ powers of (positive) divisors of $n$.
It is known from classical theory \cite[p.8]{kolberg} that:
$$\tau(8n+l) \equiv a_l \sigma_{11}(8n+l) \pmod {2^{b_l}} $$
where $gcd(l,8)=1$, $a_1 = 1$, $a_3 = 1217$, $a_5 = 1537$, $a_7 = 705$, 
$b_1 = 11$, $b_3 = 13$, $b_5 = 12$, $b_7 = 14$

We are interested in congruence class (mod $2$) of the Ramanujan function $\tau(n)$.
For $n$ odd, we deduce the following:
$$\tau(n) \equiv \sigma_{11}(n) \equiv \sum_{d | n} d^{11}
\equiv \sum_{d | n} 1 \equiv \left\lbrace
\begin{array}{ll}
1 \bmod 2 \qquad \text{if } n \text{ is a square}\\
0 \bmod 2 \qquad \text{else}\\
\end{array} \right.
$$
\subparagraph{Case $n$ even}
It is easy to calculate that $\tau(2) = -24 \equiv 0 \bmod 2$.\\
Using $\tau(p^{n+1}) = \tau(p^n)\tau(p) - p^{11}\tau(p^{n-1}) \qquad p \in \primes$ \cite[p.97]{CourseInArithmetic} with $p=2$,
it follows by induction that $\tau(2^k) \equiv 0 \bmod 2 \qquad \forall k \in \N$.\\
Using $\tau(nm) = \tau(n)\tau(m) \qquad \text{if } gcd(n,m)=1$ \cite[p.97]{CourseInArithmetic},
it follows that for all $n$ even, $\tau(n) \equiv 0 \bmod 2$.
\subparagraph[Summary]{Explicit series of the discriminant}
\label{DeltaSeries}
Therefore, the only non-zero coefficients (modulo $2$) appears on odd squares, i.e.:
$$\tau(n) \equiv \left\lbrace \begin{array}{ll}
1 \bmod 2 \qquad \text{if } n=(2m+1)^2 \quad \text{for } m \in \N\\
0 \bmod 2 \qquad \text{else} 
\end{array}\right.$$
Thus, we can write the power series of $\Delta$ as:
\[
\Delta(q) \equiv \overline{\Delta}(q) = \sum_{m=0}^{\infty} q^{(2m+1)^2} \bmod 2
\label{eq:Delta}
\]

\subsubsection{Reduced Basis}
The Miller basis for $M_k$ was obtained via the Gauss elimination of the set $\{ \Delta^jE_3^{2(d-j)+b}E_2^a | 1 \leq j \leq \dim(M_k) \}$ (with some conditions on $a$,$b$,$d$).

But $\overline{E_2^a} = \overline{E_2}^a = 1^a = 1 \bmod 2$ and similarly, $\overline{E_3^{2(d-j)+b}} = 1 \bmod 2$.
So once the above set is reduced modulo 2, we are left with $\{ \overline{\Delta}^j | 1 \leq j \leq \dim(M_k) \}$.
So the Miller basis just becomes the Gauss elimination of $\overline{\Delta}$ powers.

This is what motivates the next section.

\subsection{Space of Modular Forms Modulo Two}
We would like to have a definition for this space in a similar way as $M_k$ was used for modular forms (of weight $2k$) before reduction.

\subsubsection{Weights of Modular Forms Modulo Two}
\label{WeightModuloTwo}
We just saw that the Miller basis for $\overline{M_k}$ is (the Gaussian elimination of) $\{ \overline{\Delta}^j | 1 \leq j \leq \dim(M_k) \}$.

Now, if we look at this set not reduced modulo 2, we have:
$\{ \Delta^j | 1 \leq j \leq \dim(M_k) \}$.
This is a set of modular forms that have different weights.
However, we started with a modular forms in $M_k$, i.e. all modular forms having weight $2k$.

We understand now that modulo 2, the weight of modular form doesn't make sense any more.
This is one of the consequences of reducing modulo 2: we lose some informations about the modular forms, such as the weight.

From this observation, we should study all modular forms together, modulo 2 (instead of separating by weights).
This is why the space of modular forms modulo 2 will be denoted  $\mathcal{F}$, with mo dependence on $k$.

\subsubsection{Powers of the Modular Discriminant $\Delta$}
\paragraph{Set of Powers of the Modular Discriminant $\Delta$}
As we just saw, the Gaussian elimination of powers $\overline{\Delta}^k$ up to $\dim(\overline{M_k})$ form the Miller basis of $\overline{M_k}$ (modular forms of weight $2k$ reduced modulo 2).

To lighten the notation, we will now write $\Delta$ instead of $\overline{\Delta}$, and consider everything modulo 2.
For simplicity again, we will just take the powers of $\Delta$ to be our basis for modular forms modulo 2 (i.e. drop the Gaussian elimination process).

We define the space $\F_2[\Delta]$ in the usual way:
$$
\F_2[\Delta] = \left\lbrace \sum_{k=1}^{n}a_k \Delta^k | n \in \N, \  a_k \in \mathbb{F}_2 \right\rbrace
$$


From \ref{DeltaSeries} we had:
$$
\Delta(q) 
= \sum_{n=0}^{\infty} \tau(n)q^n
= \sum_{m=0}^{\infty} q^{(2m+1)^2}
$$
Therefore, we define 
$$
\Delta^{k}(q) 
= \sum_{n=0}^{\infty} \tau_k(n)q^n
= \left( \sum_{m=0}^{\infty} q^{(2m+1)^2} \right)^k \bmod 2
$$
Thus, we have $\tau(n)=\tau_1(n)$.

\paragraph{Proportion of zeros}
In fact, most of the coefficients $\tau_k(n)$ are $0$ modulo 2.

When $k=1$, there is already few coefficients that are ones: only the odd squares.
When raising to the $k^{th}$ power, there are even "less".

\paragraph{Conditions on non-zero coefficients}
We can find conditions on coefficients that may not be zero.

We observe:
\begin{table}[!ht]

	\begin{center}
		\begin{tabular}{|r||c|c|c|c|c|c|c|c||l|}
			\hline
			$a=$ & $0$ & $1$ & $2$ & $3$ & $4$ & $5$ & $6$ & $7$ & $\bmod 8$ \\
			\hline
			$a^2=$ & \color{BrickRed} $0$ & \color{ForestGreen} $1$ & \color{BrickRed} $4$ & \color{ForestGreen} $1$ & \color{BrickRed} $0$ & \color{ForestGreen} $1$ & \color{BrickRed} $4$ & \color{ForestGreen} $1$ & $\bmod 8$ \\
			\hline

		\end{tabular}
	\end{center}
	\caption{Squares modulo $8$}
	\label{table:SquaresMod8}
\end{table}
We remark that odd squares are all $1 \bmod 8$, and even squares are all $0 \bmod 8$.

We know from previous calculations that $\Delta(q)$ only has odd powers of $q$.
Thus, raising to the $k^{th}$ power give terms of power $n$ such that:
\begin{align*}
n &= m_1^2 + m_2^2 + m_3^2 + \dots + m_k^2 \\
&\equiv \:\; 1 \ + \:\; 1 \ + \:\; 1 \; + \dots + \:\; 1 \bmod 8 \\
&\equiv k \bmod 8
\end{align*}
Therefore: $\tau_k(n) \equiv 1 \quad \bmod 2 \implies n \equiv k \quad \bmod 8$\\
Equivalently: $n \not\equiv k \quad \bmod 8 \implies \tau_k(n) \equiv 0 \quad \bmod 2$ (by taking the contra-positive)

This means, that $\Delta^k$ may only have terms $q^n$ such that $n \equiv k \bmod 8$, i.e. $\Delta^k$ may only have terms of power congruent to $k \bmod 8$.
When $k=1$, this is that $\Delta$ may only have terms of power $1 \bmod 8$, this matches with table \ref{table:SquaresMod8}: all odd squares are $1 \bmod 8$.
\label{ObservationsMod8}

\paragraph{Even powers of $\Delta$}
We compare $\Delta^{2k}(q)$ and $\Delta^k(q^2)$:
\begin{align*}
	\Delta^{2k}(q) 
	&= \left( \sum_{m=0}^{\infty} q^{(2m+1)^2} \right)^{2k} \\
	&= \sum_{n=0}^{\infty} \#[ (2m_1+1)^2+(2m_2+1)^2+...+(2m_{2k}+1)^2=n \ | \  m_0,m_1,...,m_{2k} \in \N ] \  q^n\\
	&= \sum_{n \ even}^{\infty} \#[ (2m_1+1)^2+(2m_2+1)^2+...+(2m_k+1)^2=n/2 \ | \  m_0,m_1,...,m_k \in \N ] \  q^n\\
	&= \left( \sum_{m=0}^{\infty} q^{((2m+1)^2) \cdot 2} \right)^k \\
	&= \left( \sum_{m=0}^{\infty} (q^2)^{(2m+1)^2} \right)^k = \Delta^{k}(q^2)
\end{align*}
Thus, $\Delta^{2k}(q) = \Delta^k(q^2)$.
Therefore, we can write any modular form modulo 2 $f$ as the following:
$$
f = \sum_{s \geq 0} f_s^{2^s} \quad \text{ with } f_s \text{ having only odd powers of } \Delta
$$
\cite[(3)]{OrdreNilpotenceOperateurHecke}
So it is sufficient to study only the odd powers of $\Delta$.

\subsubsection{The Space $\mathcal{F}$}
\label{ModularFormsModTwo}
We define the space of modular forms modulo 2 denoted $\mathcal{F} $ to be \cite[2.1]{OrdreNilpotenceOperateurHecke}:
$$
\mathcal{F}
%= \left\lbrace \sum_{\substack{k \  odd,\\ k \leq n}} a_k \Delta^k | n \in \N, \  a_k \in \mathbb{F}_2 \right\rbrace
= \left\langle \Delta^k | k \text{ odd} \right\rangle
= \left\langle \Delta, \Delta^3, \Delta^5, \Delta^7, \dots \right\rangle 
$$
That is, all finite polynomials of $\Delta$ over $\F_2$, having only odd powers.
We remark that the weight of modular forms do not appear, as it was discussed before in \ref{WeightModuloTwo}.
The observations modulo 8 that we have done in \ref{ObservationsMod8} yields that it will be useful to denote:
$$
\mathcal{F}_1
= \left\langle \Delta^k \ | \ k = 1 \bmod 8 \right\rangle
= \left\langle \Delta, \Delta^9, \Delta^{17}, \Delta^{25}, \cdots \right\rangle
$$
$$
\mathcal{F}_3
= \left\langle \Delta^k \ | \ k = 3 \bmod 8 \right\rangle
= \left\langle \Delta^3, \Delta^{11}, \Delta^{19}, \Delta^{27}, \cdots \right\rangle
$$
$$
\mathcal{F}_5
= \left\langle \Delta^k \ | \ k = 5 \bmod 8 \right\rangle
= \left\langle \Delta^5, \Delta^{13}, \Delta^{21}, \Delta^{29}, \cdots \right\rangle
$$
$$
\mathcal{F}_7
= \left\langle \Delta^k \ | \ k = 7 \bmod 8 \right\rangle
= \left\langle \Delta^7, \Delta^{15}, \Delta^{23}, \Delta^{31}, \cdots \right\rangle
$$
Of course, we have:
$$
\mathcal{F} = \mathcal{F}_1 \oplus \mathcal{F}_3 \oplus \mathcal{F}_5 \oplus \mathcal{F}_7
$$
We will also introduce (as in \cite[2.]{StructureAlgebreHecke}):
$$
\mathcal{F}(n)
= \left\langle \Delta^k \ | \ k \text{ odd} \text{ and } k \leq 2n-1 \right\rangle
= \left\langle \Delta, \Delta^3, \Delta^5, \dots, \Delta^n \right\rangle
$$
This matches specifically $\overline{M_{12 \cdot n}} = \mathcal{F}(n)$.

\subsubsection{Duality between $\Delta$ and $q$}
%reference to wave-particule duality from physics
As we defined $\mathcal{F}$ above, a modular form modulo 2 is an expression of powers $\Delta^k$.
But we had from before that $\Delta = \sum_{m=0}^{\infty} q^{(2m+1)^2} \bmod 2$.
Therefore, we can translate a modular form given as a finite polynomial of $\Delta$ into an infinite polynomial of $q$.
Thus, there are two ways to write a modular form modulo 2.

This duality between the two definitions is what makes the study of modular forms modulo 2 so interesting:
we go back and forth between an infinite series and a finite polynomial.
One is easy to express, the other easy to compute.
This will lead to new reasoning.
In particular, there is a new technique of computation ("exact computations") that uses equivalence between the two ways of writing a modular form.
%use a ref to numerics, when numerics section willbe written



\subsection{Hecke Operators Modulo Two}
\subsubsection{Reduction Modulo Two}
\paragraph{Definition}
\label{DefHeckeOperatorsMod2}
Now that we have reduced modular forms modulo 2, we would like to study the Hecke operators on these reduced modular forms. We define Hecke operators modulo 2 as follows:

With $f$ a modular form modulo 2 with $q$ definition
$$
f(q) = \sum_{n \in \N} c(n)q^n
$$
we define
$$
\overline{T_p}|f(q) = \sum_{n \in \N} \gamma(n)q^n
$$
where
$$
\gamma(n) = 
\left\lbrace
\begin{array}{l l}
  c(np)        & \text{ if } p \nmid n \\
  c(np)+c(n/p) & \text{ if } p \mid  n
\end{array}
\right. 
\qquad \&\ p \text{ an odd prime}
$$

\paragraph{Well-definiteness}
We want to check that all the definitions make sense. When we look at $T_p|f$, there is a number of ways to to reduce it modulo 2: $\overline{T_p|f}$, $\overline{T_p|\overline{f}}$, $\overline{\overline{T_p}|f}$, $\overline{T_p}|\overline{f}$.

Let's compare coefficients:

$\overline{T_p|f}$:
$$
\gamma(n) 
= \sum_{a \mid (n,p),\, a \geq 1} a^{2k-1} c\left( \frac{np}{a^2} \right)
= \left\lbrace
\begin{array}{l l}
  \overline{c}(np)                   & \text{ if } p \nmid n \\
  \overline{c}(np)+\overline{c}(n/p) & \text{ if } p \mid  n
\end{array}
\right.
$$

Divisors of $(n,p)$ are $\{1\}$ or $\{1,p\}$ since $p$ is prime, so the sum split in two cases, with one or two terms.
We see now that looking at Hecke operators modulo 2 only for primes simplifies the sum to a computable formula.

As both $1$ and $p$ are odd, the term $a^{2k-1}$ reduces to $1$ modulo 2.
We understand why Hecke operators modulo 2 isn't defined for even numbers: many terms in the summation would become zero.
It would not make sense to call it a Hecke operator any more.

It also makes sense why we look at modular forms modulo 2 and not say three or five: the coefficient $a^{2k-1}$ collapse nicely modulo 2, which won't be the case modulo an other number then 2.

$\overline{T_p|\overline{f}}$:
This is (very) similar to the case before.

$\overline{T_p}|\overline{f}$:
$$
\gamma(n)
= \left\lbrace
\begin{array}{l l}
  \overline{c}(np)                   & \text{ if } p \nmid n \\
  \overline{c}(np)+\overline{c}(n/p) & \text{ if } p \mid  n
\end{array}
\right.
$$

$\overline{\overline{T_n}|f}$:
Again, this is (very) similar to the case before.

All reductions give in fact the same result, so it makes sense to reduce modular forms modulo 2, and still study the Hecke operators (but now only for odd primes).
As this all makes sense, we will now write only consider modular forms modulo 2, and we will drop the over lines for simplicity.

\subsubsection{Basic Properties}
When reduced modulo 2, Hecke operators $\overline{T_p}$ for primes $p$ have more properties then the general $T_p$.
The extra properties make the study modulo two interesting.

\paragraph{Inherited properties}
From the fact that $\overline{T_p}|f(q) = \overline{T_p|f(q)}$, we get that the Hecke operators modulo 2 keep the properties they had before being reduced.

\subparagraph{Modularity Remains}
From definition \ref{DefHeckeOperators}, a Hecke operators transform a modular form to an other.
This is because from definition, $T_nf$ is a sum of modular forms (which remain modular).
Therefore, Hecke operators modulo 2 will as well transform a modular form to an other.
This was not clear from the definition modulo 2 that we had (which was in terms of $q$ series).

\subparagraph{Commutativity}
As in general \cite[p.101]{CourseInArithmetic}:
$$
T_nT_m = T_{mn} \quad \text{ if } \gcd(m,n)=1
$$
% and $$ T(p)T(p^n)f=T(p^{n+1})f+T(p^{n-1})f $$
We get that:
$$
T_pT_q=T_qT_p \quad \forall p,q \in \primes
$$
Therefore, the Hecke operators modulo 2 commute.
This, as well, was not clear form definition.
It will be very convenient for future calculations.

\subparagraph{Linearity}
\label{HekeLinear}
From definition \ref{DefHeckeOperators}, we have that the Hecke operators are immediately linear.
That is:
$$
T_p|(f+g) = T_p|f + T_p|g
$$
(this follows directly from definition).

This property will also remain modulo 2.

\subsubsection{Nil-potency}
The properties of Hecke operators is that, given a modular form $f$, if we apply a Hecke operators enough times, the form will become zero (i.e. they are nilpotent).
The strategy to show this is to prove that for any $k$ (odd), and any prime $p$, we have:
$$
\overline{T_p}| \Delta^k = \sum_{j < k} \mu_j \Delta^j
$$
The proof of this property will be divided in two main steps:

\subparagraph{Order of $\Delta$ doesn't increase}
We first want to show that:
$$
\overline{T_p}| \Delta^k = \sum_{j \leq k} \mu_j \Delta^j
$$

From definition \ref{DefHeckeOperators}, a Hecke operators takes a modular form of weight $2k$ to an other modular form of weight $2k$.
Take a modular form modulo 2 $\overline{f}$ with degree $k$ (in terms of $\Delta$).
Now we want to know the maximum degree (again in terms of $\Delta$) of $T_p|\overline{f}$.
Let $n$ be the smallest integer such that $\overline{f} \in \overline{M_{12(2n-1)}}$.

We know $T_p|\overline{f} = \overline{T(p)f}$ and $\overline{f} \in \mathcal{F}(n) = \overline{M_{12(2n-1)}}$ so $f \in M_{12(2n-1)}$.
This implies that $T(p)f \in M_{12(2n-1)}$ so $T_p|\overline{f} = \overline{T(p)f} \in \overline{M_{12(2n-1)}} = \mathcal{F}(n)$.

Therefore, the maximum degree (in terms of $\Delta$) of $T_p|\overline{f}$ is $k$ as well.
Thus, the degree of $\overline{f}$ doesn't increase after applying a Hecke operator.

\subparagraph{Order of $\Delta$ decrease}
%Serre's "proof"
%\cite[2.3]{OrdreNilpotenceOperateurHecke}
\label{orderDecrease}
Now that we have proved that 
$$
\overline{T_p}| \Delta^k = \sum_{j \leq k} \mu_j \Delta^j
$$
we need to show that $\mu_k = 0$, so that the maximum order of $\Delta$ in fact effectively decrease.

Let's look at $\mathcal{F}(k)$ as a vector space over $\F_2$ with basis $\{ \Delta, \Delta^3, \ddots, \Delta^k \}$.
We may the represent a modular form modulo 2 by a $k$-vector over $\F_2$ (note that even powers of $\Delta$ will always be zero, but we keep track of them to lighten notation).
Then, as $T_p$ are linear (see \ref{HekeLinear}), we can represent each operator $T_p$ with a matrix.
Let $A_p$ be the $(k \times k)$-matrix (over $\F_2$) representing the action of $T_p$ on $\mathcal{F}(k)$.
Since the order of $\Delta$ doesn't increase when applying a Hecke operator, the matrix $A_p$ should be upper-triangular, i.e.:
$$
A_p = 
\begin{pmatrix}
a_{1,1} & a_{1,2} & a_{1,3} & \cdots & a_{1,k} \\
   0    & a_{2,2} & a_{2,3} & \cdots & a_{2,k} \\
   0    &    0    & a_{3,3} & \cdots & a_{3,k} \\
\vdots  & \vdots  & \vdots  & \ddots & \vdots  \\
   0    &    0    &    0    & \cdots & a_{k,k}
\end{pmatrix}
$$
We need to show that the coefficients $a_{i,i}$ are zero.

We will do this by induction.
Suppose we know $T_p$ decrease the degree of $\Delta^j$ for all $j \leq k-1$.
Translating this information to the matrix, it means that $a_{j,j}=0$ for all $j \leq k-1$.
Then, we only need to show that $a_{k,k}=0$.

Now that we have all this information on the diagonal, it makes sense to study the trace: 
$\Tr{A_p} = a_{k,k}$.
A nice interpretation of the Trace should give us an equation for $a_{k,k}$.

We can interpret the trace as the sum of eigenvaues of the matrix $A_p$, i.e. eigenvalues of the Hecke oprator $T_p$.
Some knoledge about eigenvalues of Hecke operators has been proved already (by Hatada, see \cite{EigenvaluesOfHeckeOperators}), we have:
For $p$ an odd prime, if $\lambda_p$ is an eigenvalue of $T_p$, we have the congruence: $\lambda_p \equiv 1+p \bmod 8$
Since $p$ is an odd (prime) number, we get: $\lambda_p \equiv 0 \bmod 2$.
As this is true for all eigenvalues of $T_p$, we have that the sum of eigenvalues (which corresponds to the trace of the matrix) is zero over $\F_2$.
Thus: 
%$a_{k,k} = \Tr{A_p} = \sum_{j} \lambda_{p,j} \equiv 0 \bmod 2$
$a_{k,k} = \Tr{A_p} \equiv 0 \bmod 2$

Now, this is a proof by induction, but the first case really is $k=0$, in which case, all modular forms are just $0$, so all Hecke operators are obviously zero so nilpotent in this case.

Therefore, we proved the nilpotence modulo 2 of Hecke operators $T_p$ for all $p$ odd primes.

\subsubsection{Expression as a Sum of Powers of $\Delta$}
\paragraph{Behaviour of $\mathcal{F}_i$}
Suppose $f \in \mathcal{F}_i$, using \ref{ObservationsMod8}, we have:
$$
f
= \sum_{m \equiv i \bmod 8} \mu_m\Delta^m 
= \sum_{n \equiv i \bmod 8} c(n)q^n
$$
From the definition of Hecke operator modulo 2 (\ref{DefHeckeOperatorsMod2}), we have:
$$
T_p|f
= \sum_{n \in \N} \gamma(n)q^n
\quad \text{ with }
\gamma(n) = 
\left\lbrace 
\begin{array}{l l}
c(np)        & \text{ if } p \nmid n \\
c(np)+c(n/p) & \text{ if } p \mid  n
\end{array}
\right.
$$
\begin{enumerate}
	\item [$c(np)$:] We have $np \not \equiv i \bmod 8 \implies c(np) = 0$.
	\item [$c(n/p)$:] As $p$ is an odd prime, it is an odd number, so from \ref{ObservationsMod8}, $p^2 \equiv 1 \bmod 8$, so $p^{-2} \equiv 1 \bmod 8$ as well (with $p^{-2}$ seen $\bmod 8$).
	
	Therefore, $np \not\equiv i \bmod 8 \implies  n/p \equiv \nicefrac{np}{p^2} \equiv np \not\equiv i \bmod 8 $.
	
	\item [$\gamma(n)$:] We conclude that $n \equiv np^2 \not\equiv pi \bmod 8 \implies \gamma(n) = 0$
\end{enumerate}
Using \ref{ObservationsMod8} again, we deduce that $T_p|f \in \mathcal{F}_j$ with $j \equiv pi \bmod 8$.

Overall, we have the following:
$$
f \in \mathcal{F}_i \implies T_p|f \in \mathcal{F}_j
\text{ with } j \equiv pj \bmod 8
$$

\paragraph{Application to $\Delta^k$}
As the degree of a modular form doesn't increase after applying a Hecke operator, we can apply this the modular form $\Delta^k$ to get:
$$
T_p|\Delta^k = \sum_{\substack{j \leq k \\ j \text{ odd}}} \mu_j\Delta^j
$$
As we know, moreover, that the degree of a modular form will in fact decrease, we deduce that in fact:
\[
T_p|\Delta^k = \sum_{\substack{j \leq k-2\\ j \text{ odd}}} \mu_j\Delta^j
\label{eq:TpDelta^k} \tag{*}
\]
The last observation (on $\mathcal{F}_i$), leads us to the formula:
\[
T_p|\Delta^k = \sum_{\substack{j \leq k-2\\ j \equiv pk \bmod 8}} \mu_j\Delta^j
\label{eq:TpDelta^k_bis} \tag{**}
\]
(since $\Delta^k \in \mathcal{F}_i \text{ with } k \equiv i \bmod 8$)

\subsubsection{Examples (for Small Powers of $\Delta$)}
We will describe the behaviour of Hecke operators when applied to $\Delta^k$ with $k \text{ odd}, k \leq 7$.

\paragraph{$\Delta$}
Clearly, from \eqref{eq:TpDelta^k}, we have $T_p|\Delta = 0$, since the sum is empty (for any $p$ odd prime).

\paragraph{$\Delta^3$}
From \eqref{eq:TpDelta^k}, we have $T_p|\Delta^3 = \Delta \text{ or } 0$.\\
Moreover, \eqref{eq:TpDelta^k_bis} gives $T_p|\Delta^3 = 0$ if $1 \not\equiv 3p \bmod 8$ i.e. if $p \not\equiv 3 \bmod 8$.

Now, if $p \equiv 3 \bmod 8$, we may only look at the coefficient $q^1$ of $T_p|\Delta^3$ (if it is $1$, $T_p|\Delta^3 = \Delta$ and if it is $0$, $T_p|\Delta^3 = 0$, as there is no other possibilities).

From definition (in \ref{DefHeckeOperatorsMod2}), we have that the coefficient of $q^1$ is $\gamma(1) = c(p)$ (since $p \nmid 1$) with $c$ the $q$ coefficients of $\Delta^3$.

From \eqref{eq:Delta}, the none zero coefficients of $\Delta$ are odd squares.

Now, $c(p)$ is th $p^{th}$ coefficient of $\Delta^3$. We have:
$$
\left( \Delta(q) \right)^3
= \left( \sum_{m=0}^{\infty} q^{(2m+1)^2} \right)^3
= \sum_{n=0}^{\infty} \#\{m_1, m_2, m_3 \text{ odds } | m_1^2 + m_2^2 + m_3^2 = n\} q^n
$$
So $c(p) = \#\{m_1, m_2, m_3 \text{ odds } | m_1^2 + m_2^2 + m_3^2 = p\} \bmod 2$ corresponds ($\bmod 2$) to the number of ways to write $p$ as sum of three odd squares.



Need in fact $m_1 = m_2 \neq m_3$, but then??
[I am stuck]






\paragraph{$\Delta^5$}
From \eqref{eq:TpDelta^k}, we have $T_p|\Delta^5 = \Delta^3 \text{ or } \Delta \text{ or } 0$.\\
Moreover, \eqref{eq:TpDelta^k_bis} gives:
\begin{align*}
	p \equiv 7 \bmod 8: & \quad T_p|\Delta^5 = \Delta^3 \text{ or } 0 & \text{ if } 3 \equiv 5p \bmod 8 & \quad \text{ i.e. } p \equiv 7 \bmod 8 \\
	p \equiv 5 \bmod 8: & \quad T_p|\Delta^5 = \Delta \text{ or } 0 & \text{ if } 1 \equiv 5p \bmod 8 & \quad \text{ i.e. } p \equiv 5 \bmod 8 \\
	p \equiv 1 \text{ or } 3 \bmod 8: & \quad T_p|\Delta^5 = 0 & \text{ else } &
\end{align*}

Now, if $p \equiv 7 \bmod 8$, we may only look at the coefficient $q^3$ of $T_p|\Delta^5$ (if it is $1$, $T_p|\Delta^5 = \Delta^3$ and if it is $0$, $T_p|\Delta^3 = 0$, as there is no other possibilities).

From definition (in \ref{DefHeckeOperatorsMod2}), we have that the coefficient of $q^3$ is $\gamma(3) = c(3p)$ (since $p \nmid 3$) with $c$ the $q$ coefficients of $\Delta^5$.

From \eqref{eq:Delta}, the none zero coefficients of $\Delta$ are odd squares.
Now, $c(3p)$ is th $p^{th}$ coefficient of $\Delta^5$. We have:
$$
\left( \Delta(q) \right)^5
= \left( \sum_{m=0}^{\infty} q^{(2m+1)^2} \right)^5
= \sum_{n=0}^{\infty} \#\{m_1, m_2, m_3, m_4, m_5 \text{ odds } | m_1^2 + m_2^2 + m_3^2 + m_4^2 + m_5^2 = n\} q^n
$$
So $c(3p) = \#\{m_1, m_2, m_3, m_4, m_5 \text{ odds } | m_1^2 + m_2^2 + m_3^2 + m_4^2 + m_5^2 = 3p\} \bmod 2$ corresponds ($\bmod 2$) to the number of ways to write $3p$ as sum of five odd squares.

When looked $\bmod 8$, $m_1^2 + m_2^2 + m_3^2 + m_4^2 + m_5^2 \equiv 5 \bmod 8$.
We can check, $p \equiv 7 \bmod 8$ so $3p \equiv 5 \bmod 8$.







Need in fact $m_1 = m_2 \neq m_3$, but then??
[I am stuck]






\paragraph{$\Delta^7$}

\subsubsection{Table of Hecke Operators}
Here is a table of Hecke operators that was computed with a computer:

\begin{center}
	\begin{tabular}{l|c c c c c c c c c c c c c c}
		 & $T_3$ & $T_5$ & $T_7$ & $T_{11}$ & $T_{13}$
		 & $T_{17}$ & $T_{19}$ & $T_{23}$ & $T_{29}$ & $T_{31}$ 
		 & $T_{37}$ & $T_{41}$ & $T_{43}$ & $T_{47}$ \\
		\hline
		$\Delta$ & 0 & 0 & 0 & 0 & 0 & 0 & 0 & 0 & 0 & 0 & 0 & 0 & 0 & 0\\
		$\Delta^3$ & $\Delta$ & 0 & 0 & $\Delta$ & 0 & 0 & $\Delta$ & 0 & 0 & 0 & 0 & 0 & $\Delta$ & 0\\
		$\Delta^5$ & 0 & $\Delta$ & 0 & 0 & $\Delta$ & 0 & 0 & 0 & $\Delta$ & 0 & $\Delta$ & 0 & 0 & 0\\
		$\Delta^7$ &  &  &  &  &  &  &  &  &  &  &  &  &  & \\
		$\Delta^9$ &  &  &  &  &  &  &  &  &  &  &  &  &  & \\
		$\Delta^{11}$ &  &  &  &  &  &  &  &  &  &  &  &  &  & \\
		$\Delta^{13}$ &  &  &  &  &  &  &  &  &  &  &  &  &  & \\
		$\Delta^{15}$ &  &  &  &  &  &  &  &  &  &  &  &  &  & \\
		$\Delta^{17}$ &  &  &  &  &  &  &  &  &  &  &  &  &  & \\
		$\Delta^{19}$ &  &  &  &  &  &  &  &  &  &  &  &  &  & \\
	\end{tabular}
\end{center}

It seems quite random, which makes sense since the Hecke operators depend on prime, and primes appear at random.
In line one ($\Delta^3$), we get $\nicefrac{1}{4}^{\text{th}}$ of the primes giving $\Delta$, this is a consequence of Dirichlet Density Theorem, that will be discussed later in this paper.
[Is this remark pertinent?]

\subsubsection{Nilpotent Order}
As we know that the Hecke operators are nilpotent, we may want to study the order of nil potentness.
\paragraph{Definition}
For a modular from modulo 2 $f \in \mathcal{F}$, we define the \textit{nil potentness order} to be the smallest integer $g(f)$ such that we have 
$$
T_{p_1} T_{p_2} \cdots T_{p_{g(f)}} | f = 0
$$
for any set of primes numbers $p_1, p_2, \dots, p_{g(f)} \in \primes$.
The primes $p_i$ involved do not need to be distinct.
Note as well that from commutativity of the Hecke operators, the order of the primes $p_i$ doesn't matter.

By convention, we write $g(0)= -\infty$.
With a slight abuse of notation, we will write $g(k)$ for $g(\Delta^k)$.

\paragraph{Properties}
\subparagraph{Well-definiteness}
All Hecke operators lower by at least two the maximum degree of $\Delta$ in the $\Delta$-expansion of a modular form modulo 2 \ref{orderDecrease}.
We deduce that $g(f) \leq g(T_p|f) + 1$.
Applied to $Delta^k$, we get: $g(k) \leq g(k-2) + 1$.
Therefore, by induction, we have $g(k) \leq \floor{\frac{k+1}{2}}$.
This implies by the same occasion, the well definiteness of the order of nil potentness for all modular form modulo two.

\subparagraph{conjecture}
Let $f_1,f_2 \in \mathcal{F}$ be modular form modulo 2, such that
$f_1 = \Delta^k + \sum_{j < k} \mu_j \Delta^k$
and
$f_2 = \Delta^k + \sum_{j < k} \nu_j \Delta^k$
(i.e. both having maximum power $\Delta^k$).

Then $g(f_1)=g(f_2)$.

\subparagraph{proof attempt}
By induction?\\
Case $k=1$: This is trivial.\\
Case $k=3$: This is rather straightforward as well: $g(\Delta^3)=1$ and $g(\Delta^3+\Delta)=1$.

Now, suppose this is true for all $k^* < k$.
Want to show it is true for $k$.

Need 
$$
T_p|\Delta^k = \Delta^l + O(\Delta^{l-2}) 
\implies
\exists q \in \primes \text{ such that } 
T_q|\Delta^{k+2}=\Delta^m + O(\Delta^{l-2}) 
\text{ with } m \geq l
$$
------ or ------
$$
MaxOrder(T_p|\Delta^k) \geq MaxOrder(T_p|\Delta^k+\Delta^l)
\quad \forall l<k
$$
Is any of the two reasonable to prove?


Note that if it is not provable, I should do a numerical analysis, and mension it as a [rather strong, depending on the analysis] conjecture.
Note that it might be wrong, and maybe that numerical analysis will find it out. In such a case, it is nice to mention.

\paragraph{Examples}
\subparagraph{By hand}
We can compute a few nil potentness by hand:
\begin{itemize}
	\item $g(0) = -\infty$
	
	\item $g(\Delta) = 1$:\\
	$
	T_p(\Delta) = 0
	$
	as order of $\Delta$ decrease, see \ref{orderDecrease}
	
	\item $g(\Delta^3) = 2$:\\
	$
	T_p|\Delta^3 = \Delta \text{ or } 0
	$\\
	thus:
	$
	g(\Delta^3) = 1 + \max(g(\Delta), g(0)) = 2
	$
	
	\item $g(\Delta^3+\Delta) = 2$\\
	similarly
	
	\item $g(\Delta^5) = 2$:\\
	$
	T_p|\Delta^5 = \Delta \text{ or } 0
	$\\
	thus:
	$
	g(\Delta^5) = 1 + \max(g(\Delta), g(0)) = 2
	$
	
	\item $g(\Delta^5+\Delta^3+\Delta) = g(\Delta^5+\Delta^3) = g(\Delta^5+\Delta) = 2$\\
	similarly
	
	\item $g(\Delta^7) = 3$:\\
	$
	T_p(\Delta^7) = \Delta^5 \text{ or } \Delta^3 \text{ or }  \Delta \text{ or } 0
	$\\
	thus:
	$
	g(\Delta^7) = 1 + \max(g(\Delta^5), g(\Delta^3), g(\Delta), g(0)) = 3
	$
\end{itemize}



\subparagraph{Computer calculated}
We will look at $\Delta^{95}$...


