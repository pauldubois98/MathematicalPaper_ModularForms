% !TeX spellcheck = en_GB
\section{Frobenian Elements}
\subsection{Context}
Let $R$ be a commutative ring, $M$ and $P$ ideals in $R$. We can then prove the followings:
\begin{theorem}
    \begin{itemize}
        \item $M \text{ is maximal} \iff R/M \text{ is a field}$
        \item $P \text{ is prime} \iff R/P \text{ is an integral domain}$
    \end{itemize}
\end{theorem}
\begin{property}
    Maximal ideals are prime. 
\end{property}
\begin{proof}
    \begin{align*}
        M \text{ maximal ideal} &\iff R/M \text{ field}\\
                                &\implies \text{ R/M Integral Domain } \iff M \text{ prime ideal}
    \end{align*}
\end{proof}

If $L/K$ is a Galois extension, then we will denote it's Galois group by $\Gal{L/K}$.



-----------



Let $K$ be a number field, and $\mathcal{O}_K$ be the corresponding ring of integers.
Let $\mathfrak{p}$ be a non-zero prime ideal in $\mathcal{O}_K$.
Let $L/K$ be a finite extension and again, $\mathcal{O}_L$ be the ring of integers in $L$.
Then we know that $\mathcal{O}_L$ is the integral closure of $\mathcal{O}_K$ in $L$
We have $\mathfrak{p}\mathcal{O}_L$ an ideal in $\mathcal{O}_L$.
It is not a prime ideal in general, but as $L/K$ is finite, there exists a factorization as the following:
$$
\mathfrak{p}\mathcal{O}_L = \prod_{i=1}^r \mathfrak{P}_i^{e_i}.
$$
Where the integers $e_i$ are called the ramification indexes.
We also have $\mathfrak{P}_i \cap \mathcal{O}_K = \mathfrak{p}$, and we say that the ideals $\mathfrak{P}_i$ in $L$ extend the ideal $\mathfrak{p}$ in $K$.

Then, there are three possibilities for an ideal: it may split, ramify of be inert.
\begin{definition}[Ideal Ramifies]
    We say that an ideal $\mathfrak{p}$ ramifies in $L/K$ if a ramification index $e_i$ is greater then one, 
    i.e. if $e_i>1$ for some $1 \leq i \leq r$.
\end{definition}
\begin{definition}[Ideal Splits]
    We say that $\mathfrak{p}$ splits in $L/K$ if none of the ramification indexes $e_i$ is greater then one, and $r$ is a least two; 
    i.e. if $e_i=1 \quad \forall 1 \leq i \leq r$ and $r \geq 2$.
\end{definition}
\begin{definition}[Ideal Inert]
    We say that $\mathfrak{p}$ is inert in $L/K$ if there is only one ramification index $e_1$ and it is equal to one;
    i.e. if $e_1=1$ and $r=1$.
\end{definition}
We know that the extension $L/K$ is ramified in the primes that divide the discriminant. Therefore, the extension is unramified in all but finitely many prime ideals.


\subsection{Residue Fields Extensions}
The ideal $\mathfrak{p}$ defines the \textit{residue field} $F=\mathcal{O}_K/\mathfrak{p}$.
The ideals $\mathfrak{P}_i$ define the \textit{residue fields} $F_i=\mathcal{O}_L/\mathfrak{P}_i$.\\
The field $F$ then naturally embeds to $F_i$ (so each $\mathfrak{P}_i$ defines a field extension).
The \textit{inertia degree} of $\mathfrak{P}_i$ is the degree $f_i=[F_i:F]=[\mathcal{O}_L/\mathfrak{P}_i:\mathcal{O}_K/\mathfrak{p}]$ of this extension.
We then observe that $[L:K] = \sum_{i=1}^r e_if_i$
We can then specify when an ideal splits or ramifies completely:
\begin{definition}[Ideal Splits Completely]
    We say that $\mathfrak{p}$ splits completely in $L/K$ if all ramification indexes $e_i$ and inertia degrees $f_i$ are one.
    i.e. if $e_i=f_i=1 \quad \forall 1 \leq i \leq r$.\\
    In this case, $r=[L:K]$.
\end{definition}
\begin{definition}[Ideal Ramifies Completely]
    We say that $\mathfrak{p}$ ramifies completely in $L/K$ if the inertia degrees $f_1$ is one, and $r$ is one.
    i.e. if $r=1$ and $f_1=1$.\\
    In this case, $e_1=[L:K]$.
\end{definition}

\subsection{Norms of Ideals}
We define the \textit{norm of an ideal} $I$ in $\mathcal{O}_K$ as $N(I)=|\mathcal{O}_K/I|$.
If $\mathfrak{p} \subset \mathcal{O}_K$ is a prime ideal, then we can put $(p)=\mathfrak{p} \cap \Z$.
It follows that $p\mathcal{O}_K \subset \mathfrak{p}$.
$\mathcal{O}_K$ is a free $\Z-module$ of rank $[K:\Q]=q$, i.e. $\exists \alpha_1,\dots,\alpha_q \text{ s.t. } \mathcal{O}_K=\Z \alpha_1 \oplus \cdots \oplus \Z \alpha_q$.
Thus, $|\mathcal{O}_K/\mathfrak{p}| \leq |\mathcal{O}_K/(p)| \leq p^q$.
We have $\Norm{\mathfrak{p}} = |\mathcal{O}_K/\mathfrak{p}|=p^m$ and $\Norm[L/\Q]{\mathfrak{P}_i} = \Norm[K/\Q]{\mathfrak{p}}^{f_i}$.
This implies $\Norm{\mathfrak{P}_i} = |\mathcal{O}_L/\mathfrak{P}_i| = p^{mf_i}$.\\
We also have: 
$\mathcal{O}_K/\mathfrak{p} \cong \mathbb{F}_{\Norm{\mathfrak{p}}}$ and $\mathcal{O}_L/\mathfrak{P}_i \cong \mathbb{F}_{\Norm{\mathfrak{P}_i}}$.

\subsection{Galois Extensions Simplifications}
When the extension $L/K$ is Galois, the ramification indexes $e_i$ are all the same ($e_i=e$), as well as the inertia degrees $f_i=f$.
We then have
$$
\mathfrak{p}\mathcal{O}_L = \prod_{i=1}^r \mathfrak{P}_i^{e}
\text{ and } [L:K] = ref.
$$
The Galois group $\Gal{L/K}$ is often denoted $G$.

We define the \textit{decomposition group} $G_{\mathfrak{P}}$ of the ideal $\mathfrak{P}$ to be $\{\sigma \in G \mid \sigma(\mathfrak{P})=\mathfrak{P} \}$.
It turns out that 
$G_{\mathfrak{P}} 
\cong \Gal{\nicefrac{\mathcal{O}_L/\mathfrak{P}}{\mathcal{O}_K/\mathfrak{p}}}
\cong \Gal{ \mathbb{F}_{p^{mf}}/\mathbb{F}_{p^{f}}}$.
Moreover, it is a cyclic group, so $G_{\mathfrak{P}} = <\tilde{\sigma}>$.

\subsection{Unramified Prime Simplifications}
When the ideal $\mathfrak{p}$ is unramified, $e=1$, so we get:
$$
\mathfrak{p}\mathcal{O}_L 
= \prod_{i=1}^r \mathfrak{P} \text{ and } [L:K] = rf
$$

\subsection{The Frobenius Element}
\subsubsection{Definition}
We can construct the \textit{Frobenius element} (sometimes also called the \textit{Artin symbol}, or the \textit{Frobenius map}) that depend on the extension $L/K$ and ideal $\mathfrak{P}$ in $\mathcal{O}_L$.
It is denoted $\Frob{L/K}{\mathfrak{P}}$, and is \textit{the} element $\sigma \in G$ such that:
$$
\sigma \mathfrak{P} = \mathfrak{P}
\quad \text{ and } \quad
\sigma(\alpha) \equiv \alpha^{\Norm[K/\Q]{\mathfrak{p}}} \bmod{\mathfrak{P}} \quad \forall \alpha \in \mathcal{O}_L.
$$

The second condition is the interesting one; while the first is only useful to make the Frobenius element unique.
The second condition defines a unique element only up to conjugacy class.
Most of the time, we will consider abelian extensions, so the conjugacy classes will only have one element, and the first condition will be dropped.

We define the \textit{Frobenius element} for $\mathfrak{p}$ (denoted $\Frob{L/K}{\mathfrak{p}}$) in a meaning full manner, to be the set
$$
\{\Frob{L/K}{\mathfrak{P}} | \mathfrak{P} \text{ extending } \mathfrak{p}\} \subset G
.$$

The following properties imply that $\Frob{L/K}{\mathfrak{p}}$ is in fact a conjugacy class in the Galois group $\Gal{L/K}$.
Hence, we refer to $\Frob{L/K}{\mathfrak{p}}$ as the Frobenius conjugacy class.

\begin{property}
	If $\tau \in G$, then 
	$\Frob{L/K}{\mathfrak{\tau P}} = \tau \Frob{L/K}{\mathfrak{P}} \tau^{-1}$.
\end{property}
\begin{proof}
	For all $x \in \mathcal{O}_L$, we have:
	$$
	\Frob{L/K}{\mathfrak{P}}x 
	= x^{\Norm[K/\Q]{\mathfrak{p}}} \mod \mathfrak{P}
	$$
	But all such $x$ may be written as $\tau^{-1}(x)$, so we have:
	$$
	\Frob{L/K}{\mathfrak{P}} \tau^{-1}(x) 
	= {(\tau^{-1} x)}^{\Norm[K/\Q]{\mathfrak{p}}} \mod \mathfrak{P}.
	$$
	Which gives:
	$$
	\tau \Frob{L/K}{\mathfrak{P}}\tau^{-1}(x) 
	= x^{\Norm[K/\Q]{\mathfrak{p}}} \mod \mathfrak{P}.
	$$
\end{proof}


\begin{property}
	If $\mathfrak{P}_1$ and $\mathfrak{P}_2$ extend $\mathfrak{p}$, then $\Frob{L/K}{\mathfrak{P}_1}$ and $\Frob{L/K}{\mathfrak{P}_2}$ are conjugates.
\end{property}
\begin{proof}
	We have the following scheme:\\
	\begin{tikzpicture}[text width=10cm, align=flush center]
	%nodes
	\node (L)                                           {$L$};
	
	\node (subsetL)[right of=L, node distance=0.75cm]   {$\supseteq$};
	\node (P1)[right of=subsetL, node distance=0.75cm]  {$\mathfrak{P}_1$};
	\node (P2)[right of=P1, node distance=1cm]          {$\mathfrak{P}_2$};
	
	\node (K) [below of=L, node distance=2cm]           {$K$};
	\node (subsetK)[right of=K, node distance=1cm]      {$\supseteq$};
	\node (p) [right of=subsetK, node distance=1cm]     {$\mathfrak{p}$};
	
	%links
	\draw[-] (L) to node {} (K);
	\draw[-] (p) to node {} (P1);
	\draw[-] (p) to node {} (P2);
	\end{tikzpicture}
	
	There is an element $\tau \in G$ such that $\tau(\mathfrak{P}_1)=\mathfrak{P}_2$.
	Then using last property, we deduce that $\Frob{L/K}{\mathfrak{P}_1}$ and $\Frob{L/K}{\mathfrak{P}_2}$ are conjugates.
\end{proof}

Never the less, is important to notice at this point that if $L/K$ is an abelian extension (i.e. $G$ is abelian), then every conjugacy class in $\Gal{L/K}$ are made up of only one element.
In this case, we sometimes use $\Frob{L/K}{\mathfrak{p}}$ to denote the Frobenius element $\Frob{L/K}{\mathfrak{P}}$, where $\mathfrak{P}$ is any prime lying above $\mathfrak{p}$.



\subsubsection{Examples}
\paragraph{$\Q[\sqrt{7}]/\Q$ (quadratic field extension)}
\label{QuadraticExtensionExample}
%Looking at $\Q[\sqrt{7}]:\Q$ (which is a Galois extension).
We have minimum polynomial $m(x)=x^2-7$, the discriminant is $\Delta = 4.7 = 28$.\\
We write
$$
G = \Gal{\Q[\sqrt{7}]:\Q} = <\sigma \mid \sigma^2 = 1_G> \cong C_2.
$$
As $C_2$ is abelian, we will have no problem defining Frobenius elements.

\subparagraph{The prime ideal $(3)$}
%We look at the prime ideal $(3)$:
As $m(x) = (x+1)(x-1) \mod 3$, we have $(3) = (3, \sqrt{7}+1)(3, \sqrt{7}-1)$.\\
As well, $\Norm[{\Q[\sqrt{7}]/\Q}]{(3)} = 3$ and $\Norm[{\Q[\sqrt{7}]/\Q}]{(3, \sqrt{7}+1)} 
=\Norm[{\Q[\sqrt{7}]/\Q}]{(3, \sqrt{7}-1)} 
= 3
$, but $\Norm[\Q/\Q]{(3)} = 3$.
So we have:
\begin{align*}
    \Frob{\Q[\sqrt{7}]:\Q}{(3, \sqrt{7}+1)}:
    \alpha   &\to \alpha^{\Norm[\Q/\Q]{(3)}} \bmod (3, \sqrt{7}+1)\\
    \sqrt{7} &\to \left( \sqrt{7} \right)^3 \equiv \sqrt{7} \bmod (3, \sqrt{7}+1)
\end{align*}
Thus, $\Frob{\Q[\sqrt{7}]:\Q}{(3, \sqrt{7}+1)} = 1_G \in G$.
Similarly, $\Frob{\Q[\sqrt{7}]:\Q}{(3, \sqrt{7}-1)} = 1_G \in G$.

\subparagraph{The prime ideal $(5)$}
%We look at the prime ideal $(5)$:
As $m(x)$ has no root $\bmod 2$. So $m(x)$ is irreducible $\bmod 5$ and $(5)$ is inert in $\Q[\sqrt{7}]$
As well, $\Norm[{\Q[\sqrt{7}]/\Q}]{(5)} = 5^2 = 25$ but $\Norm[\Q/\Q]{(5)} = 5$.
So we have:
\begin{align*}
    \Frob{\Q[\sqrt{7}]:\Q}{(5)}: 
    \alpha   &\to \alpha^{\Norm[\Q/\Q]{(5)}} \bmod (5)\\
    \sqrt{7} &\to \left( \sqrt{7} \right)^5 \equiv -\sqrt{7} \bmod (5)
\end{align*}
Thus, $\Frob{\Q[\sqrt{7}]:\Q}{(5)} = \sigma \in G$.

\paragraph{$\Q[\zeta_n]/\Q$ ($n^{th}$ Cyclotomic Field Extensions)}
We have minimum polynomial:
$$
\Phi(x)= \prod_{\substack{1 \leq k \leq n \\ \gcd(k,n)=1}} 
\left( x-e^{2 i \pi \frac{k}{n}} \right)
$$
(so degree of the extension is $\varphi(n)$, where $\varphi$ is Euler totient function).
Discriminant of the extension is:
$$
\Delta = (-1)^{\varphi(n)/2}
\frac{n^{\varphi(n)}}{\prod_{p \mid n} p^{\varphi(n)/(p-1)}};
$$
see \cite[Proposition 2.7]{IntroductionToCyclotomicFields}.

The Galois group $G$ consist of $\sigma_k$ such that $\sigma_k(\zeta_n^i)=\zeta_n^{ik}$, with $\gcd(k,n)=1$.)
Note as well that $G$ is abelian, so it is simple to calculate the Frobenius element.
It is straightforward that $G$ is naturally isomorphic to the multiplicative group $\left( \nicefrac{\Z}{n\Z} \right)^\times$.
Note that $\sigma \in G$ is determined by $\sigma(\zeta_n)$.
Note as well that this group is abelian.

With $p \in \primes$, a prime that is unramified in $\Q[\zeta_n]/\Q$, let $P$ be an ideal lying above $(p)$.
We want to look at $\Frob{\Q[\zeta_n]/\Q}{P}$.
We have:
\begin{align*}
	\Frob{\Q[\zeta_n]/\Q}{P}: 
	\alpha&\to \alpha^{\Norm[\Q/\Q]{(p)}} \bmod P\\
	\zeta_n&\to \zeta_n^p \bmod P
\end{align*}

\paragraph{Case $\Q[\zeta_{10}]/\Q$ ($10^{th}$ cyclotomic field extension)}
We denote by $\zeta_{10}=e^{\pi i/5}$ the $10^{th}$ root of unity.
We have minimum polynomial $m(x) = x^4-x^3+x^2-x+1$ (so degree of the extension is 4), the discriminant is $\Delta = 5^3$.

We write $
G = \Gal{\Q[\zeta_{10}]:\Q} 
= < \sigma: \zeta_{10} \to \zeta_{10}^3 \mid \sigma^4 = Id > \cong C_4$.

%\begin{center}
%	\begin{tabular}{r|l}
%		$x$ & $x^4-x^3+x^2-x+1$ \\
%		\hline
%		$-5$ & $781 = 11 \cdot 71$ \\
%		$-4$ & $341 = 11 \cdot 31$\\
%		$-3$ & $121 = 11^2$\\
%		$-2$ & $31$\\
%		$-1$ & $5$\\
%		$0$  & $1$\\
%		$1$  & $1$\\
%		$2$  & $11$\\
%		$3$  & $61$\\
%		$4$  & $205 = 5 \cdot 41$\\
%		$5$  & $521$\\		
%	\end{tabular}
%\end{center}



\subparagraph{The prime ideal $(3)$}
%We look at the prime ideal $(3)$:
As $m(x)$ has no root $\bmod 3$, so $(3)$ is inert.
We have:
\begin{align*}
	\Frob{\Q[\zeta_{10}]/\Q}{(3)}:
	\alpha   &\to \alpha^{\Norm[\Q/\Q]{(3)}} \bmod (3)\\
	\zeta_{10} &\to \left( \zeta_{10} \right)^3 \bmod (3)
\end{align*}
Thus, $\Frob{\Q[\zeta_{10}]:\Q}{(3)} = \sigma \in G$.

\subparagraph{The prime ideal $(7)$}
%We look at the prime ideal $(3)$:
As $m(x)$ has no root $\bmod 7$, so $(7)$ is inert.
We have:
\begin{align*}
	\Frob{\Q[\zeta_{10}]/\Q}{(7)}:
	\alpha   &\to \alpha^{\Norm[\Q/\Q]{(7)}} \bmod (7)\\
	\zeta_{10} &\to \left( \zeta_{10} \right)^7 \bmod (7)
\end{align*}
Thus, $\Frob{\Q[\zeta_{10}]:\Q}{(7)} = \sigma^3 \in G$.

\subparagraph{The prime ideal $(11)$}
As $m(x) = (x-2)(x+3)(x+4)(x+4) \bmod 11$, so $(11)$ splits.
We have:
\begin{align*}
	\Frob{\Q[\zeta_{10}]/\Q}{(11)}:
	\alpha   &\to \alpha^{\Norm[\Q/\Q]{(11)}} \bmod (11)\\
	\zeta_{10} &\to \left( \zeta_{10} \right)^{11} = \zeta_{10} \bmod (11)
\end{align*}
Thus, $\Frob{\Q[\zeta_{10}]:\Q}{(11)} = \sigma^4 = Id \in G$.



\subsubsection{Behaviour in Towers of Fields}
We will consider the following scheme:\\
\begin{tikzpicture}[text width=10cm, align=flush center, node distance=0.75cm]
%nodes

\node (M)                                      {$M$};
\node (subsetM)[left of=M, node distance=0.5cm]  {$\subset$};
\node (OM)[left of=subsetM, node distance=0.5cm] {$\mathcal{O}_M$};
\node (supsetM)[right of=M]                    {$\supseteq$};
\node (PM)[right of=supsetM]                   {$\mathfrak{P}$};


\node (L) [below of=M, node distance=2cm]      {$L$};
\node (subsetL)[left of=L, node distance=0.5cm]  {$\subset$};
\node (OL)[left of=subsetL, node distance=0.5cm] {$\mathcal{O}_L$};
\node (supsetL)[right of=L]                    {$\supseteq$};
\node (PL)[right of=supsetL]                    {$\mathfrak{p}$};


\node (K) [below of=L, node distance=2cm]        {$K$};
\node (subsetK)[left of=K, node distance=0.5cm]  {$\subset$};
\node (OK)[left of=subsetK, node distance=0.5cm] {$\mathcal{O}_K$};
\node (supsetK)[right of=K]                    {$\supseteq$};
\node (PK)[right of=supsetK]                    {$p$};


%links
\draw[-] (M) to node {} (L);
\draw[-] (L) to node {} (K);
\draw[-] (PM) to node {} (PL);
\draw[-] (PL) to node {} (PK);
\end{tikzpicture}

In such a situation, we can define (for $M/K$ Galois)
$\Frob{M/K}{\mathfrak{P}}$, 
$\Frob{M/K}{p}$, 
$\Frob{M/L}{\mathfrak{P}}$, 
$\Frob{M/L}{\mathfrak{p}}$.
If, in addition, $L/K$ is normal, we can as well define:
$\Frob{L/K}{\mathfrak{p}}$ , and 
$\Frob{L/K}{p}$ (see \cite[p.99]{AlgebraicNumberFields}).
We will look at properties of these Frobenius elements (relation between each others).

\begin{property}
	$$
	\Frob{M/K}{\mathfrak{P}}^{f(\mathfrak{P}/\mathfrak{p})} = \Frob{M/L}{\mathfrak{P}}
	$$
	
	[what is this $f(\mathfrak{P}/\mathfrak{p})$? is it the fields of $\mathfrak{P}$ over $\mathfrak{p}$?]
\end{property}
\begin{proof}
[to write...]
\cite[p.99]{AlgebraicNumberFields}
\end{proof}

\begin{property}
	$$
	\Frob{L/K}{\mathfrak{p}} = \left. \Frob{M/K}{\mathfrak{P}} \right|_L
	$$
\end{property}
\begin{proof}
	Let $\sigma = \Frob{M/K}{\mathfrak{P}} \in \Gal{M/K}$ so 
	$\sigma: M \to M \text{ s.t. } \left. \sigma \right|_K = Id \text{ and } \sigma \text{ is an autotomorphism}$.\\
	Similarly, let $\tau = \Frob{L/K}{\mathfrak{p}} \in \Gal{L/K}$ so 
	$\tau: L \to L \text{ s.t. } \left. \tau \right|_K = Id \text{ and } \tau \text{ is an autotomorphism}$.
	
	As $M$ extends $L$, $\sigma$ being an automorphism of $M$ makes it an automorphism of $L$ as well.
	The restriction condition stays the same.
	% done with Djordjo
	% \cite[p.99]{AlgebraicNumberFields}
\end{proof}


\begin{property}
	$$
	\Gal{L/K} \cong \nicefrac{\Gal{M/K}}{\Gal{M/L}}
	$$
\end{property}
\begin{proof}
	Let $\sigma \in \Gal{M/K}$, i.e. $\sigma: M \to M \text{ s.t. } \left. \sigma \right|_K = Id \text{ and } \sigma \text{ is an autotomorphism}$.
	
	Let $\phi: \Gal{M/K} \to \Gal{L/K}$ be such that:
	$\phi(\sigma) = \left. \sigma \right|_L$.
	This is well defined as an automorphism of $M$ restricts to an automorphism of $L$ when $M$ extends $L$.
	
	It is trivial to check that $\phi$ is a homomorphism.
	
	The kernel of $\phi$ is clearly $\Gal{M/L}$.
	
	The image of $\phi$ is $\Gal{L/K}$ as every element of $\Gal{L/K}$ may be extended to $\Gal{M/K}$.
	
	Therefore, the property follows via the $1^{st}$ isomorphism theorem.
	% done with Djordjo
\end{proof}

\begin{property}
	We have: 
	$$
	\mathfrak{p} \text{ splits complitely in } L 
	\iff \Frob{L/K}{\mathfrak{P}} = 1
	$$
\end{property}
\cite[p.100]{AlgebraicNumberFields}


We will consider the following scheme:\\
\begin{tikzpicture}[text width=10cm, align=flush center, node distance=2cm]
%nodes
\node (M) {$L_1L_2$};
\node (supsetM)[right of=M, node distance=0.75cm] {$\supseteq$};
\node (PM)[right of=supsetM, node distance=0.5cm] {$\mathfrak{P}$};

\node (L1) [below of=M, left of=M] {$L_1$};
\node (supsetL1)[right of=L1, node distance=0.5cm] {$\supseteq$};
\node (PL1)[right of=supsetL1, node distance=0.5cm] {$\mathfrak{p}_1$};

\node (L2) [below of=M, right of=M] {$L_2$};
\node (supsetL2)[right of=L2, node distance=0.5cm] {$\supseteq$};
\node (PL2)[right of=supsetL2, node distance=0.5cm] {$\mathfrak{p}_2$};

\node (K) [below of=M, node distance = 4cm] {$K$};
\node (supsetK)[right of=K, node distance=0.5cm] {$\supseteq$};
\node (PK)[right of=supsetK, node distance=0.5cm] {$\mathfrak{p}$};


%links
\draw[-] (M) to node {} (L1);
\draw[-] (M) to node {} (L2);
\draw[-] (L1) to node {} (K);
\draw[-] (L2) to node {} (K);
\end{tikzpicture}



\begin{property}
	We have:
	$$
	\Frob{L_1L_2/K}{\mathfrak{P}}
	= \Frob{L_1/K}{\mathfrak{p}_1} \times \Frob{L_2/K}{\mathfrak{p}_2}
	$$
\end{property}
\cite[p.100]{AlgebraicNumberFields}

\begin{property}
	We have: 
	$$
	\mathfrak{p} \text{ splits complitely in } L_1L_2
	\iff
	\mathfrak{p} \text{ splits complitely in } L_1 \text{ and } L_2
	$$
\end{property}
\begin{proof}
	Combine the last two proposition.
	\cite[p.100]{AlgebraicNumberFields}
\end{proof}





\subsection{The Chebotarev's Density Theorem}
\subsubsection{Motivations}
\label{DensityMotivation}
If we look at the distribution of primes numbers modulo a number (15 in the next example), we get a table as follows:

Table mod 15:
\begin{center}
	\begin{tabular}{r|l}
		$\bmod 15$ & primes (up to 500)\\
		\hline
		0& \\
		1& 31, 61, 151, 181, 211, 241, 271, 331, 421, \\
		2& 2, 17, 47, 107, 137, 167, 197, 227, 257, 317, 347, 467, \\
		3& 3, \\
		4& 19, 79, 109, 139, 199, 229, 349, 379, 409, 439, 499, \\
		5& 5, \\
		6& \\
		7& 7, 37, 67, 97, 127, 157, 277, 307, 337, 367, 397, 457, 487, \\
		8& 23, 53, 83, 113, 173, 233, 263, 293, 353, 383, 443, \\
		9& \\
		10& \\
		11& 11, 41, 71, 101, 131, 191, 251, 281, 311, 401, 431, 461, 491, \\
		12& \\
		13& 13, 43, 73, 103, 163, 193, 223, 283, 313, 373, 433, 463, \\
		14& 29, 59, 89, 149, 179, 239, 269, 359, 389, 419, 449, 479, \\
	\end{tabular}
\end{center}
%[table display to improve!!!]
It looks like there are classes of primes.
We would like to characterize this repartition: that is, decide if classes are finite or infinite, and quantify the repartitions.



\subsubsection{Notions of Density}
As discussed previously, we are interested in subsets of $\primes$ (the set of primes numbers).
Euler proved that there are infinitely many primes.
%cite Euler?
Therefore, there are two types of subsets of $\primes$: the ones that are infinite, and the finites ones.
For finite sets, we can characterise the size by just counting elements.
In fact, we will mainly be interested in sets that have infinitely many primes, and again, we would like a notion of size.

A suitable way would be to compare the subset with the set of all primes, and, say look at the proportions of primes included in the subset.
We call this the density, there are two rigorous ways to define it:
\begin{definition}[Natural density]
	We say that $S \subseteq \primes$ has natural density $\delta$ when:
	$$
	\lim_{x \to +\infty}
	\frac{ \# \{ p \in \primes, p < x \mid p \in S \}}
	{ \# \{ p \in \primes, p < x \mid p \in \primes \}} = \delta
	$$
\end{definition}
\begin{definition}[Analytic density or Dirichlet density]
	We say that $S \subseteq \primes$ has analytical (or Dirichlet) density $\delta$ when:
	$$
	\lim_{s \to 1^+}
	\left( \sum_{p \in S} \frac{1}{p^s} \right) 
	\left( \sum_{p \in \primes} \frac{1}{p} \right)^{-1} = \delta
	$$
\end{definition}

Note that the natural density may not exist.
However, when both exist, the two densities are the same.
%cite!



\subsubsection{Statement}
One of the most important results that use Frobenian maps is probably the Chebotarev density theorem.
\begin{theorem}[Chebotarev Density Theorem]
	With $L/K$ an extension of Galois group $G=\Gal{L/K}$.\\
	Let $C$ be a conjugacy class in $G$.
	
	Then, the proportion of unramified primes ideals $\mathfrak{p}$ in $K$ that have Frobenius element $\Frob{L/K}{\mathfrak{p}}=C$ \footnote{When depending on a prime in the "lower" field, the Frobenius element is a conjugacy class to be well defined.} is $\nicefrac{|C|}{|G|}$.
\end{theorem}
We see that Frobenius elements are in the heart of this theorem.
It was proved by Nikolai Chebotarev in his thesis (\cite{ChebotarevTheorem}).

\subsubsection{Example}
We go through an example of Chebotarev Density Theorem for an extension of order 3.
We look at $K/\Q$ with $K \cong \nicefrac{\Q[x]}{(x^3-3x-1)}$ (i.e. the number field with defining polynomial $x^3 - 3x - 1$).
Using SageMath \footnote{See the code in appendix, \ref{code:ChebotarevExample}}, we have:
The discriminant of this extension is $3^4=81$, and the extension is Galois.
We define $G = \Gal{K/\Q}$ the Galois group of the extension, and we have $G \cong C_3 = \left\langle \sigma \mid \sigma^2 = 1 \right\rangle $ since the order of the extension is 3).

Then, an unramified prime in $\Q$ may remain irreducible in $K/\Q$, split in $K/\Q$.
If $p$ splits in $K\Q$, then $\Frob{K/\Q}{p}=1$ (the identity of the Galois group $G$).
If $p$ remains inert in $K/\Q$, then $\Frob{K/\Q}{p}=\sigma \text{ or } \sigma^2$.
As the discriminant is finite, there are finitely many primes that ramifies.
Applying the Chebotarev Density theorem: one third of the primes will split in this extension, and two third will remain inert.

\subsubsection{Special Case}
Here, we want to apply Chebotarev theorem in the case of a quadratic field extension.
We are looking at the field extension $L/K = \Q[\sqrt{d}]/\Q$ for $d \in \Z$ a square-free integer.
Denote by $G = \Gal{\Q[\sqrt{d}]/\Q} \cong C_2$ the Galois group of this extension.
This group is abelian (so all conjugacy classes are made of a single element), and for any conjugacy class $C$, $\nicefrac{|C|}{|G|} = \nicefrac{1}{2}$.
Now, for a prime $p$ unramified, we want to calculate the Frobenius element.
If $p$ is unramified, either $p \mathcal{O}_{\Q[\sqrt{d}]} = R_1R_2$ or $p \mathcal{O}_{\Q[\sqrt{d}]} = R$.

In the first case, we have $\left( \frac{d}{p} \right) = 1$ (i.e. $\sqrt{d} \in \F_p$, so $d$ is a square modulo $p$).
In this case, $\sqrt{d}^p \equiv \sqrt{d} \bmod p$ so $\Frob{\Q[\sqrt{d}]}{p} = \left\lbrace Id: \sqrt{d} \to \sqrt{d} \right\rbrace \in G$.

In the second case, $\left( \frac{d}{p} \right) = -1$ (i.e. $\sqrt{d} \not\in \F_p$, so $d$ is not a square modulo $p$).
In this case, $\sqrt{d}^p \not\equiv \sqrt{d} \bmod p$ as there is no other choice, $\Frob{\Q[\sqrt{d}]}{p} = \left\lbrace \sigma: \sqrt{d} \to -\sqrt{d} \right\rbrace \in G$.

Then by Chebotarev's density theorem, we have that the density of primes $p$ such that $\left( \frac{d}{p} \right) = \pm1$ is $\nicefrac{1}{2}$ in both cases.
Therefore, we have the following summary:
\begin{center}
	\begin{tabular}{|c|c|}
		\hline
		Primes $p \in \primes$ such that: & Density:\\
		\hline
		$\left( \frac{d}{p} \right) = +1$ & $\nicefrac{1}{2}$\\
		$\left( \frac{d}{p} \right) =  0$ & $0$\\
		$\left( \frac{d}{p} \right) = -1$ & $\nicefrac{1}{2}$\\
		\hline
	\end{tabular}
\end{center}
Thus, for a square free $d$, $\left( \frac{d}{p} \right)$ is as often $+1$ as $-1$ (for a prime $p$), and  $\left( \frac{d}{p} \right) = 0$ happens only finitely many times.



\subsection{The Dirichlet's Density Theorem}
\subsubsection{Statement}
The most common application of Chebotarev density theorem is probably the Dirichlet's density theorem.
\begin{theorem}[Dirichlet's Density Theorem]
	Let $n \in \N^*$, $a \in \N$ such that $\gcd(a,n) = 1$. 
	If $S = \{ p \in \primes \mid p \equiv a \mod n \}$, then $S$ has density $\nicefrac{1}{\varphi(n)}$.
\end{theorem}

\subsubsection{Link with Chebotarev}
This is a direct application of Chebotarev's density theorem for the field extension $\Q[\zeta]:\Q$ where $\zeta$ is the $n^{th}$ root of unity (this is the cyclotomic field).
The Galois group is abelian (it is precisely $G=\Z_n^{\times}$ and has order $\varphi(n)$).
The abelian property implies that all conjugacy classes are made of a single element.
Thus, for any conjugacy class $C$, the fraction $\nicefrac{|C|}{|G|}$ is just $\nicefrac{1}{\varphi(n)}$.
Primes ideals in $\Q$ are just primes numbers.
Therefore, Chebotarev gives Dirichlet's density theorem in the particular case of cyclotomic extensions.

\subsubsection{Example}
% We have:% $\Phi(x) = x^8 - x^7 + x^5 - x^4 + x^3 - x + 1$
% $\Delta = 3^4 5^6 = 1265625$.
% We consider $\Q[\zeta_{15}/\Q]/\Q$.
% The Galois group $G = \left( \nicefrac{\Z}{15\Z} \right)^\times$ has elements:
% $\alpha \to \alpha$, $\alpha \to \alpha^2$, $\alpha \to \alpha^4$, $\alpha \to \alpha^7$, $\alpha \to \alpha^8$, $\alpha \to \alpha^{11}$, $\alpha \to \alpha^{13}$, $\alpha \to \alpha^{14}$.
Here, look at the example of Dirichlet theorem in the case $n=15$ from the motivation subsection above (see \ref{DensityMotivation}).
We apply the last theorem in the case of $n=15$: $\varphi(15)=8$.
We define $S_k = \{ p \in \primes \mid p \equiv k \bmod 15 \}$.
By Dirichlet density theorem,  the density of $S_k$ is $\nicefrac{1}{8}$ if $k$ and $15$ are co-prime (i.e. if $k = 1,2,4,7,8,11,13,14$), otherwise (if $k=0,3,5,6,9,10,12$) it is $0$.
This is what we could conjecture from the observations.

\section{Frobenian Maps and Governing Fields}
\subsection{Frobenian Maps}
\paragraph{Class functions}
Let $G$ be a group, $\Omega$ a set, and $f: G \to \Omega$.
We say that $f$ is a \textit{class function} (of $G$) if $f$ is constant on conjugacy classes of $G$.
That is, if $f$ remains unchanged under conjugation map of $G$.

\paragraph{$S$-Frobenian Maps}
This definition is taken from \cite[§3.3]{LecturesOnN_Xp}.
Let $K$ be a number field. Let $P$ be the set of primes ideals in $K$.
Let $S \subseteq P$ be a subset of primes ideal of $K$.
We say that a function $f: P \setminus S \to \Omega$ is $S$-Frobenian if there exists an $M$, extending $K$ and a class function $\phi: \Gal{M/K} \to \Omega$ such that $f = \phi \circ \Frob{M/K}{}$, i.e. $f(\mathfrak{p})= \phi \circ \Frob{M/K}{\mathfrak{p}} \quad \forall \mathfrak{p} \in P$ \footnote{Note that $\phi(\Frob{M/K}{\mathfrak{p}})$ is well defined since $\phi$ is a class function, and $\Frob{M/K}{\mathfrak{p}}$ is a conjugacy class of $\Gal{M/K}$.}.

\paragraph{Frobenian Maps}
With the same setting as above, $f: P \setminus S \to \Omega$ is Frobenian if there exists a finite set $S \subset P$ such that $f$ is $S$-Frobenian.\\
In general, we will take $S$ to be the set of ramified primes (there are finitely many, since they divide the Discriminant, which is finite).
In the case of $K=\Q$, the set of primes ideals becomes just the set of primes $\primes$.
And a map $f: \primes \to \Omega$ is said to be \textit{Frobenian} if there exists a field extension $M/\Q$ and a class function $\phi:\Gal{M/\Q} \to \Omega$ such that for all but finitely many (all unramified) primes $p \in \primes$, we have $f(p)=\phi(\Frob{M/\Q}{p})$.

\paragraph{$a_{ij}(p)$ Frobenian}

We recall that for all $p$ odd prime, 
$$
T_p = \sum_{i,j \geq 0} a_{ij}(p)T_3^iT_5^j
\qquad \text{ with } a_{ij}(p) \in \F_2.
$$
\begin{theorem}\cite[§7]{StructureAlgebreHecke}.
	For $i$ and $j$ fixed, the map $p \to a_{ij}(p)$ is Frobenian.\\
	That is, for all $i,j \geq 0$, there exists an extension $M_{ij}/\Q$ and a class function $\phi_{ij}: \Gal{M_{ij}/\Q} \to \F_2$ such that $a_{ij}(p)=\phi_{ij}(\Frob{M_{ij}/\Q}{p})$ for all $p \in \primes$ unramified in $M_{ij}/\Q$.
	
	Moreover, $M_{ij}/\Q$ is a finite Galois extension, unramified for odd primes.
\end{theorem}
In such a configuration, $M_{ij}$ are called \textit{governing fields}.

%% to add later, maybe
%\paragraph{Link with Theorem from Bellaïche}
%Let $A = \F_2\left[ T_3, T_5 \right]$, and $\text{Frac}(A)=\{\nicefrac{f}{g} | f,g \in A, g \neq 0 \}$.
%With $k$ a field, we define 
%$
%\SL{2}{k} = \left\lbrace 
%\begin{pmatrix}
%	a & b \\
%	c & d
%\end{pmatrix} | \ a,b,c,d \in k, ad-bc \neq 0
%\right\rbrace
%$.
%
%We define $G_{\Q} = \Gal{\overline{\Q}/\Q}$, the profinite group.
%Where $\overline{\Q}$ is the algebraic closure of $\Q$, i.e. the composition of all number fields $K/\Q$.
%%number field: finite extension of $\Q$
%
%% what is $G_{\Q}$?
%Let $\mathcal{K}$ be the set of all Galois number fields.
%With $A_K = \Gal{K/\Q}$, $(A_K)_{K \in \mathcal{K}}$ is a family of groups.
%There are groups homomorphisms $f_{ij}$
%
%[...bla...]


\subsection{Governing Fields}
It is nice to know that the maps $p \to a_{ij}(p)$ are Frobenius, but to compute $a_{ij}(p)$, we need to know explicitly the governing field (which will depend on $i$ and $j$).
\subsubsection{Basics}
\paragraph{Notations}
We denote by $M_{ij}$ \textit{a} governing field of the map $p \to a_{ij}(p)$.
From the theorem above, we know that such a field exist.
Note that such a governing fields may not be unique (and in fact we will prove next that it is never unique).

\paragraph{Properties}
Using properties of Frobenius elements, we have that if $L$ extends $M_{ij}$, then $L$ will also be a governing fields for the map $p \to a_{ij}(p)$.
This implies that governing fields aren't unique.

We also have:
\begin{itemize}
	\item $T_p \in \F_2[[x^2, y^2]] \text{ if } p \equiv 1 \bmod 8$
	\item $T_p \in x.\F_2[[x^2, y^2]] \text{ if } p \equiv 3 \bmod 8$
	\item $T_p \in y.\F_2[[x^2, y^2]] \text{ if } p \equiv 5 \bmod 8$
	\item $T_p \in xy.\F_2[[x^2, y^2]] \text{ if } p \equiv 7 \bmod 8$
\end{itemize}
\cite[§7]{StructureAlgebreHecke}
[proof??]

\paragraph{Examples}
Here are expansions of $T_p$ in series of $T_3^aT_5^b$ for primes $p<20$:\\
$T_{3} = T_3^{1}T_5^{0}$\\
$T_{5} = T_3^{0}T_5^{1}$\\
$T_{7} = T_3^{1}T_5^{1} + T_3^{3}T_5^{1} + T_3^{3}T_5^{3} + T_3^{5}T_5^{1} + T_3^{1}T_5^{7} + T_3^{1}T_5^{9} + T_3^{7}T_5^{3} + T_3^{7}T_5^{5} + T_3^{9}T_5^{3} + T_3^{11}T_5^{1} + T_3^{3}T_5^{11} + T_3^{5}T_5^{9} + T_3^{13}T_5^{1} + T_3^{3}T_5^{13} + T_3^{5}T_5^{11} + T_3^{9}T_5^{7} + T_3^{11}T_5^{5} + T_3^{13}T_5^{3} + T_3^{3}T_5^{15} + T_3^{7}T_5^{11} + T_3^{9}T_5^{9} + T_3^{13}T_5^{5} + T_3^{15}T_5^{3} + \dots $\\   
$T_{11} = T_3^{1}T_5^{0} + T_3^{1}T_5^{2} + T_3^{3}T_5^{0} + T_3^{1}T_5^{4} + T_3^{3}T_5^{2} + T_3^{5}T_5^{0} + T_3^{1}T_5^{6} + T_3^{3}T_5^{4} + T_3^{7}T_5^{2} + T_3^{1}T_5^{10} + T_3^{3}T_5^{8} + T_3^{7}T_5^{4} + T_3^{9}T_5^{2} + T_3^{11}T_5^{2} + T_3^{3}T_5^{12} + T_3^{5}T_5^{10} + T_3^{7}T_5^{8} + T_3^{11}T_5^{4} + T_3^{13}T_5^{2} + T_3^{9}T_5^{8} + T_3^{17}T_5^{0} + \dots $\\
$T_{13} = T_3^{0}T_5^{1} + T_3^{0}T_5^{3} + T_3^{2}T_5^{1} + T_3^{0}T_5^{5} + T_3^{4}T_5^{1} + T_3^{2}T_5^{5} + T_3^{4}T_5^{3} + T_3^{6}T_5^{1} + T_3^{0}T_5^{9} + T_3^{2}T_5^{7} + T_3^{6}T_5^{3} + T_3^{0}T_5^{11} + T_3^{6}T_5^{5} + T_3^{8}T_5^{3} + T_3^{10}T_5^{1} + T_3^{2}T_5^{11} + T_3^{4}T_5^{9} + T_3^{6}T_5^{7} + T_3^{10}T_5^{3} + T_3^{2}T_5^{13} + T_3^{4}T_5^{11} + T_3^{14}T_5^{1} + T_3^{2}T_5^{15} + T_3^{4}T_5^{13} + T_3^{6}T_5^{11} + T_3^{12}T_5^{5} + T_3^{16}T_5^{1} + \dots $\\
$T_{17} = T_3^{0}T_5^{2} + T_3^{2}T_5^{0} + T_3^{2}T_5^{2} + T_3^{0}T_5^{6} + T_3^{4}T_5^{2} + T_3^{6}T_5^{0} + T_3^{2}T_5^{6} + T_3^{4}T_5^{4} + T_3^{6}T_5^{2} + T_3^{10}T_5^{0} + T_3^{2}T_5^{10} + T_3^{4}T_5^{8} + T_3^{6}T_5^{6} + T_3^{10}T_5^{2} + T_3^{2}T_5^{12} + T_3^{6}T_5^{8} + T_3^{10}T_5^{4} + T_3^{2}T_5^{14} + T_3^{6}T_5^{10} + T_3^{8}T_5^{8} + T_3^{12}T_5^{4} + T_3^{14}T_5^{2} + T_3^{4}T_5^{14} + T_3^{8}T_5^{10} + T_3^{10}T_5^{8} + T_3^{12}T_5^{6} + T_3^{16}T_5^{2} + T_3^{18}T_5^{0} + \dots $\\
$T_{19} = T_3^{1}T_5^{0} + T_3^{3}T_5^{0} + T_3^{1}T_5^{4} + T_3^{3}T_5^{2} + T_3^{1}T_5^{6} + T_3^{5}T_5^{2} + T_3^{3}T_5^{6} + T_3^{7}T_5^{2} + T_3^{9}T_5^{0} + T_3^{1}T_5^{10} + T_3^{7}T_5^{4} + T_3^{9}T_5^{2} + T_3^{11}T_5^{0} + T_3^{1}T_5^{12} + T_3^{5}T_5^{8} + T_3^{11}T_5^{2} + T_3^{13}T_5^{0} + T_3^{3}T_5^{12} + T_3^{7}T_5^{8} + T_3^{9}T_5^{6} + T_3^{11}T_5^{4} + T_3^{13}T_5^{2} + T_3^{3}T_5^{14} + T_3^{7}T_5^{10} + T_3^{11}T_5^{6} + T_3^{15}T_5^{2} + T_3^{17}T_5^{0} + \dots $\\
Expansions for larger primes may be found in \ref{expansionsOfTp}.



\subsubsection{Known Governing Fields}
For $i$,$j$ small, we can compute explicitly the maps $a_{ij}$.
\paragraph{Known $a_{ij}(p)$}
We have:
\begin{itemize}
	\item $a_{10}(p)=1 \equiv p \equiv 3 \bmod 8$
	\item $a_{01}(p)=1 \equiv p \equiv 5 \bmod 8$
	\item $a_{11}(p)=1 \equiv p \equiv 7 \bmod 8$
	\item $a_{20}(p)=1 \equiv \exists a,b \in \Z \text{ and } b \text{ odd},\text{ such that } p=a^2+8b^2, p \equiv 3 \bmod 8$
	\item $a_{02}(p)=1 \equiv \exists a,b \in \Z \text{ and } b \text{ odd},\text{ such that } p=a^2+16b^2, p \equiv 3 \bmod 8$
\end{itemize}
\cite[§7]{StructureAlgebreHecke}
[proof??]

\paragraph{Corresponding Governing Fields}
We then have the following corresponding governing fields:
\begin{itemize}
	\item $M_{10} = \Q(\zeta_8)$ with $\zeta_8$ the $8^{th}$ root of unity
	\item $M_{01} = \Q(\zeta_8)$
	\item $M_{11} = \Q(\zeta_8, \sqrt{\zeta_8}) = \Q(\zeta_{16})$ with $\zeta_{16}$ the $16^{th}$ root of unity.
	\item $M_{10} = \Q(\zeta_8, \sqrt{1+i})$
	\item $M_{10} = \Q(\zeta_8, \sqrt[4]{2})$
\end{itemize}
\cite[§7]{StructureAlgebreHecke}
[proof??]




\subsubsection{Research of Governing Fields}
[ NEW TOPIC !!! ]








