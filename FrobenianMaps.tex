\section{Class Field Theory}
\subsection{Context}
Let $R$ be a commutative ring, $M$ and $P$ ideals in $R$. We can then prove the followings:
\begin{theorem}
    \begin{itemize}
        \item $M \text{ is maximal} \iff R/M \text{ is a field}$
        \item $P \text{ is prime} \iff R/P \text{ is an integral domain}$
    \end{itemize}
\end{theorem}
\begin{property}
    Maximal ideals are prime. 
\end{property}
\begin{proof}
    \begin{align*}
        M \text{ maximal ideal} &\iff R/M \text{ field}\\
                                &\implies \text{ R/M Integral Domain } \iff M \text{ prime ideal}
    \end{align*}
\end{proof}

If $L/K$ is a Galois extension, then we will denote it's Galois group by $\Gal{L/K}$.



-----------



Let $K$ be a number field, and $\mathcal{O}_K$ be the corresponding ring of integers.
Let $\mathfrak{p}$ be a non-zero prime ideal in $\mathcal{O}_K$.
Let $L/K$ be a finite extension and again, $\mathcal{O}_L$ be the ring of integers in $L$.

Then we know that $\mathcal{O}_L$ is the integral closure of $\mathcal{O}_K$ in $L$

We have $\mathfrak{p}\mathcal{O}_L$ an ideal in $\mathcal{O}_L$.
It is not a prime ideal in general, but as $L/K$ is finite, there exists a factorization as the following:
$$
\mathfrak{p}\mathcal{O}_L = \prod_{i=1}^r \mathfrak{P}_i^{e_i}
$$
Where the integers $e_i$ are called the ramification indexes.
We also have $\mathfrak{P}_i \cap \mathcal{O}_K = \mathfrak{p}$, and we say that the ideals $\mathfrak{P}_i$ in $L$ extend the ideal $\mathfrak{p}$ in $K$.

Then, there are three possibilities for an ideal: it may split, ramifies of be inert.
\begin{definition}[Ideal Ramifies]
    We say that an ideal $\mathfrak{p}$ ramifies in $L/K$ if a ramification index $e_i$ is greater then one, 
    i.e. if $e_i>1$ for some $1 \leq i \leq r$.
\end{definition}
\begin{definition}[Ideal Splits]
    We say that $\mathfrak{p}$ splits in $L/K$ if none of the ramification indexes $e_i$ is greater then one, and $r$ is a least two; 
    i.e. if $e_i=1 \quad \forall 1 \leq i \leq r$ and $r \geq 2$.
\end{definition}
\begin{definition}[Ideal Inert]
    We say that $\mathfrak{p}$ is inert in $L/K$ if there is only one ramification index $e_1$ and it is equal to one;
    i.e. if $e_1=1$ and $r=1$.
\end{definition}
We know that the extension $L/K$ is ramified in the primes that divide the discriminant. Therefore, the extension is unramified in all but finitely many prime ideals.


\subsection{Residue Fields Extensions}
The ideal $\mathfrak{p}$ defines the residue field $F=\mathcal{O}_K/\mathfrak{p}$.
The ideals $\mathfrak{P}_i$ define the residue fields $F_i=\mathcal{O}_L/\mathfrak{P}_i$.\\
The field $F$ then naturally embeds to $F_i$ (so each $\mathfrak{P}_i$ defines a field extension).
The inertia degree of $\mathfrak{P}_i$ is the degree $f_i=[F_i:F]=[\mathcal{O}_L/\mathfrak{P}_i:\mathcal{O}_K/\mathfrak{p}]$ of this extension.

We then observe that $[L:K] = \sum_{i=1}^r e_if_i$

We can then specify when an ideal splits or ramifies completely.
\begin{definition}[Ideal Splits Completely]
    We say that $\mathfrak{p}$ splits completely in $L/K$ if all ramification indexes $e_i$ and inertia degrees $f_i$ are one.
    i.e. if $e_i=f_i=1 \quad \forall 1 \leq i \leq r$.\\
    In this case, $r=[L:K]$.
\end{definition}
\begin{definition}[Ideal Ramifies Completely]
    We say that $\mathfrak{p}$ ramifies completely in $L/K$ if the inertia degrees $f_1$ is one, and $r$ is one.
    i.e. if $r=1$ and $f_1=1$.\\
    In this case, $e_1=[L:K]$.
\end{definition}

\subsection{Norms of Ideals}
We define the norm of an ideal $I$ in $\mathcal{O}_K$ as $N(I)=|\mathcal{O}_K/I|$.

If $\mathfrak{p} \subset \mathcal{O}_K$ is a prime ideal, then we can put $(p)=\mathfrak{p} \cap \Z$.
It follows that $p\mathcal{O}_K \subset \mathfrak{p}$.
$\mathcal{O}_K$ is a free $\Z-module$ of rank $[K:\Q]=q$, i.e. $\exists \alpha_1,\dots,\alpha_q \text{ s.t. } \mathcal{O}_K=\Z \alpha_1 \oplus \cdots \oplus \Z \alpha_q$.
Thus, $|\mathcal{O}_K/\mathfrak{p}| \leq |\mathcal{O}_K/(p)| \leq p^q$.

We have $Norm(\mathfrak{p})=|\mathcal{O}_K/\mathfrak{p}|=p^m$
and $Norm_{L/\Q}(\mathfrak{P}_i) = Norm_{K/\Q}(\mathfrak{p})^{f_i}$.
This implies $Norm(\mathfrak{P}_i)=|\mathcal{O}_L/\mathfrak{P}_i|=p^{mf_i}$.\\
We also have: $\mathcal{O}_K/\mathfrak{p} \cong \mathbb{F}_{Norm(\mathfrak{p})}$
and $\mathcal{O}_L/\mathfrak{P}_i \cong \mathbb{F}_{Norm(\mathfrak{P}_i)}$

\subsection{Galois Extensions Simplifications}
When the extension $L/K$ is Galois, the ramification indexes $e_i$ are all the same ($e_i=e$), as well as the inertia degrees $f_i=f$.
We then have $$\mathfrak{p}\mathcal{O}_L = \prod_{i=1}^r \mathfrak{P}_i^{e} \text{ and } [L:K] = ref$$
The Galois group $\Gal{L/K}$ is often denoted $G$

We define the decomposition group $G_{\mathfrak{P}}$ of the ideal $\mathfrak{P}$ to be $\{\sigma \in G | \sigma(\mathfrak{P})=\mathfrak{P} \}$.
It turns out that $G_{\mathfrak{P}} \cong \Gal{\nicefrac{\mathcal{O}_L/\mathfrak{P}}{\mathcal{O}_K/\mathfrak{p}}} \cong \Gal{ \mathbb{F}_{p^{mf}}/\mathbb{F}_{p^{f}}}$.
Moreover, it is a cyclic group, so $G_{\mathfrak{P}} = <\tilde{\sigma}>$.

\subsection{Unramified Prime Simplifications}
When the ideal $\mathfrak{p}$ is unramified, $e=1$, so we get:
$$\mathfrak{p}\mathcal{O}_L = \prod_{i=1}^r \mathfrak{P} \text{ and } [L:K] = rf$$

\subsection{The Frobenius Element}
\subsubsection{Definition}
[is it ok a a subsection title? (perhaps too generic?) alternatives: well definiteness, introduction,  ]

We can construct the Frobenius element (sometimes also called the Artin symbol, or the Frobenius map) that depend on the extension $L/K$ and ideal $\mathfrak{P}$ in $\mathcal{O}_L$.
It is denoted $\Frob{L/K}{\mathfrak{P}}$, and is \textit{the} element $\sigma \in G$ such that:
$$
\sigma(\alpha) \equiv \alpha^{Norm(\mathfrak{p})} \bmod{\mathfrak{P}} \quad \forall \alpha \in \mathcal{O}_L.
$$

At the moment, we can not talk of the Frobenius element as it is not unique: it is unique only up to a conjugacy class.
This is not too pathological, as Frobenius element of primes ideals $\mathfrak{P}_1$ and $\mathfrak{P}_2$ that extend the same prime $\mathfrak{p}$ are conjugates.

We define the Frobenius element for $\mathfrak{p}$ (denoted $\Frob{L/K}{\mathfrak{p}}$) in a meaning full manner, to be the set $\{ \Frob{L/K}{\mathfrak{P}} | \mathfrak{P} \text{ extending } \mathfrak{p} \} \subset G$.

[is this construction / definition ok? or maybe I should rephrase it? (it look weird to me)]

\begin{property}
	If $\tau \in G$, then 
	$\Frob{L/K}{\mathfrak{\tau P}} = \tau \Frob{L/K}{\mathfrak{P}} \tau^{-1}$.
\end{property}
\begin{proof}
	For all $x \in \mathcal{O}_L$, we have:
	$$
	\Frob{L/K}{\mathfrak{P}}x 
	= x^{Norm(\mathfrak{p})} \mod \mathfrak{P}
	$$
	But all such $x$ may be written as $\tau^{-1}(x)$, so we have:
	$$
	\Frob{L/K}{\mathfrak{P}} \tau^{-1}(x) 
	= {(\tau^{-1} x)}^{Norm(\mathfrak{p})} \mod \mathfrak{P}
	$$
	Which gives:
	$$
	\tau \Frob{L/K}{\mathfrak{P}}\tau^{-1}(x) 
	= x^{Norm(\mathfrak{p})} \mod \mathfrak{P}
	$$
\end{proof}


\begin{property}
	If $\mathfrak{P}_1$ and $\mathfrak{P}_2$ extend $\mathfrak{p}$, then $\Frob{L/K}{\mathfrak{P}_1}$ and $\Frob{L/K}{\mathfrak{P}_2}$ are conjugates.
\end{property}
\begin{proof}
	We have the following scheme:\\
	\begin{tikzpicture}[text width=10cm, align=flush center]
	%nodes
	\node (L)                                           {$L$};
	
	\node (subsetL)[right of=L, node distance=0.75cm]   {$\supseteq$};
	\node (P1)[right of=subsetL, node distance=0.75cm]  {$\mathfrak{P}_1$};
	\node (P2)[right of=P1, node distance=1cm]          {$\mathfrak{P}_2$};
	
	\node (K) [below of=L, node distance=2cm]           {$K$};
	\node (subsetK)[right of=K, node distance=1cm]      {$\supseteq$};
	\node (p) [right of=subsetK, node distance=1cm]     {$\mathfrak{p}$};
	
	%links
	\draw[-] (L) to node {} (K);
	\draw[-] (p) to node {} (P1);
	\draw[-] (p) to node {} (P2);
	\end{tikzpicture}
	
	There is an element $\tau \in G$ such that $\tau(\mathfrak{P}_1)=\mathfrak{P}_2$.
	Then using last property, we deduce that $\Frob{L/K}{\mathfrak{P}_1}$ and $\Frob{L/K}{\mathfrak{P}_2}$ are conjugates.
\end{proof}

Never the less, is important to notice at this point that if $G$ is abelian, then all conjugacy classes are made up of only one element.
Therefore, the Frobenius element is well defined in this case.
Moreover, it will only depend on the prime $\mathfrak{p}$ that is extended.
[should this comment be before or after the property?]



\subsubsection{Examples}
\paragraph{Case $\Q[\sqrt{7}]:\Q$ (quadratic field extension)}
%Looking at $\Q[\sqrt{7}]:\Q$ (which is a Galois extension).
We have minimum polynomial $m(x)=x^2-7$, the discriminant is $\Delta = 4.7 = 28$.\\
We write $G = \Gal{\Q[\sqrt{7}]:\Q} = <\sigma \ | \ \sigma^2 = 1_G> \cong C_2$.

\subparagraph{The prime ideal $(2)$}
%We look at the prime ideal $(2)$:
As $m(x) = (x+1)^2 \mod 2$, we have $(2) = (2, \sqrt{7}+1)^2$.\\
As well, $Norm((2)) = 2^2 = 4$ and $Norm((2, \sqrt{7}+1)) = 2$.

So we have:
\begin{align*}
    \Frob{\Q[\sqrt{7}]:\Q}{(2, \sqrt{7}+1)}: \alpha &\to \alpha^{Norm((2))} \ (3, \sqrt{7}+1)\\
    \sqrt{7} &\to [\sqrt{7}]^4 \ (2, \sqrt{7}+1) = -\sqrt{7} \ (2, \sqrt{7}+1) 
\end{align*}
Thus, $\Frob{\Q[\sqrt{7}]:\Q}{(2, \sqrt{7}+1)} = \sigma \in G$.

\subparagraph{The prime ideal $(3)$}
%We look at the prime ideal $(3)$:
As $m(x) = (x+1)(x-1) \mod 3$, we have $(3) = (3, \sqrt{7}+1)(3, \sqrt{7}-1)$.\\
As well, $Norm((3)) = 3^2 = 9$ and $Norm((3, \sqrt{7}+1)) = Norm((3, \sqrt{7}-1)) = 3$.

So we have:
\begin{align*}
    \Frob{\Q[\sqrt{7}]:\Q}{(3, \sqrt{7}+1)}: \alpha&\to \alpha^{Norm((3))} \ (3, \sqrt{7}+1)\\
    \sqrt{7}&\to [\sqrt{7}]^9 \ (3, \sqrt{7}+1) = \sqrt{7} \ (3, \sqrt{7}+1)
\end{align*}
Thus, $\Frob{\Q[\sqrt{7}]:\Q}{(3, \sqrt{7}+1)} = 1_G \in G$.
Similarly, $\Frob{\Q[\sqrt{7}]:\Q}{(3, \sqrt{7}-1)} = 1_G \in G$.

\paragraph{Case $\Q[\zeta_6]:\Q$ ($6^{th}$ cyclotomic field extension)}
%Looking at $\Q[\zeta_6]:\Q$ (which is a Galois extension).
We have minimum polynomial $m(x)=x^2-x+1$ (so the degree of the extension is 2), the discriminant is $\Delta = -3$ (here, $\zeta_6=e^{\pi i/3}$ denotes the $6^{th}$ root of unity).

We write $G = \Gal{\Q[\zeta_6]:\Q} = < \sigma: \zeta_6 \to \zeta_6^2, \tau: \zeta_6 \to \zeta_6^3 \ | \ \sigma^3=\tau^2=Id, \sigma\tau=\tau\sigma > \cong C_2 \times C_3$.
Note that $g \in G$ is determined by $g(\zeta_6)$.
Note as well that $G$ is abelian, so the Frobenius element is well defined.

\subparagraph{The prime ideal $(3)$}
%We look at the prime ideal $(3)$:
As $m(x) = (x+1)^2 \mod 3$, we have $(3) = (3, \zeta_6+1)^2$.\\
As well, $Norm((3)) = 3^2 = 9$ and $Norm((3, \zeta_6+1)) = Norm((3, \zeta_6-1)) = 3$.

So we have:
\begin{align*}
    \Frob{\Q[\zeta_6]:\Q}{(3, \zeta_6+1)}: \alpha & \to \alpha^{Norm((3))} \ (3, \zeta_6+1)\\
    \zeta_6   & \to [\zeta_6]^9 \ (3, \zeta_6+1) = -1 \ (3, \zeta_6+1) = \zeta_6 \ (3, \zeta_6+1)
\end{align*}
Thus, $\Frob{\Q[\zeta_6]:\Q}{(3, \zeta_6+1)} = Id \in G$.

\subparagraph{The prime ideal $(7)$}
%We look at the prime ideal $(7)$:
As $m(x) = (x+2)(x-3) \mod 3$, we have $(7) = (7, \zeta_6+2)(7, \zeta_6-3)$.\\
As well, $Norm((7)) = 7^2 = 49$ and $Norm((7, \zeta_6+2)) = Norm((3, \zeta_6-3)) = 7$.

So we have:
\begin{align*}
    \Frob{\Q[\zeta_6]:\Q}{(7, \zeta_6+2)}: \alpha & \to \alpha^{Norm((7))} \ (7, \zeta_6+2)\\
    \zeta_6   & \to [\zeta_6]^49 \ (7, \zeta_6+2) = \zeta_6 \ (7, \zeta_6+2)
\end{align*}
Thus, $\Frob{\Q[\zeta_6]:\Q}{(7, \zeta_6+2)} = Id \in G$.
Similarly, $\Frob{\Q[\zeta_6]:\Q}{(7, \zeta_6-3)} = Id \in G$.

\subparagraph{The prime ideal $(13)$}
%We look at the prime ideal $(7)$:
As $m(x) = (x+3)(x-4) \mod 3$, we have $(13) = (13, \zeta_6+3)(13, \zeta_6-4)$.\\
As well, $Norm((13)) = 13^2 = 169$ and $Norm((13, \zeta_6+3)) = Norm((3, \zeta_6-4)) = 13$.

So we have:
\begin{align*}
    \Frob{\Q[\zeta_6]:\Q}{(13, \zeta_6+3)}: \alpha & \to \alpha^{Norm((13))} \ (13, \zeta_6+3)\\
    \zeta_6   & \to [\zeta_6]^169 \ (13, \zeta_6+3) = \zeta_6 \ (13, \zeta_6+3)
\end{align*}
Thus, $\Frob{\Q[\zeta_6]:\Q}{(13, \zeta_6+3)} = Id \in G$.
?? always identity?? [to be solved]



\subsubsection{Behaviour in Chained Extensions}

We will consider the following scheme:\\
\begin{tikzpicture}[text width=10cm, align=flush center, node distance=0.75cm]
%nodes

\node (M)                                      {$M$};
\node (subsetM)[left of=M, node distance=0.5cm]  {$\subset$};
\node (OM)[left of=subsetM, node distance=0.5cm] {$\mathcal{O}_M$};
\node (supsetM)[right of=M]                    {$\supseteq$};
\node (PM)[right of=supsetM]                   {$\mathfrak{P}$};


\node (L) [below of=M, node distance=2cm]      {$L$};
\node (subsetL)[left of=L, node distance=0.5cm]  {$\subset$};
\node (OL)[left of=subsetL, node distance=0.5cm] {$\mathcal{O}_L$};
\node (supsetL)[right of=L]                    {$\supseteq$};
\node (PL)[right of=supsetL]                    {$\mathfrak{p}$};


\node (K) [below of=L, node distance=2cm]        {$K$};
\node (subsetK)[left of=K, node distance=0.5cm]  {$\subset$};
\node (OK)[left of=subsetK, node distance=0.5cm] {$\mathcal{O}_K$};
\node (supsetK)[right of=K]                    {$\supseteq$};
\node (PK)[right of=supsetK]                    {$p$};


%links
\draw[-] (M) to node {} (L);
\draw[-] (L) to node {} (K);
\draw[-] (PM) to node {} (PL);
\draw[-] (PL) to node {} (PK);
\end{tikzpicture}

In such a situation, we can define (for $M/K$ Galois)
$\Frob{M/K}{\mathfrak{P}}$, 
$\Frob{M/K}{p}$, 
$\Frob{M/L}{\mathfrak{P}}$, 
$\Frob{M/L}{\mathfrak{p}}$.
If, in addition, $L/K$ is normal:
$\Frob{L/K}{\mathfrak{p}}$ , and 
$\Frob{L/K}{p}$.
\cite[p.99]{AlgebraicNumberFields}

We will look at properties of these Frobenius elements (relation between each others).

\begin{property}
	$$
	\Frob{M/K}{\mathfrak{P}}^{f(\mathfrak{P}/\mathfrak{p})} = \Frob{M/L}{\mathfrak{P}}
	$$
	
	[what is this $f(\mathfrak{P}/\mathfrak{p})$? is it the fields of $\mathfrak{P}$ over $\mathfrak{p}$?]
\end{property}
\begin{proof}
[to write...]
\cite[p.99]{AlgebraicNumberFields}
\end{proof}

\begin{property}
	$$
	\Frob{L/K}{\mathfrak{p}} = \left. \Frob{M/K}{\mathfrak{P}} \right|_L
	$$
\end{property}
\begin{proof}
	Let $\sigma = \Frob{M/K}{\mathfrak{P}} \in \Gal{M/K}$ so 
	$\sigma: M \to M \text{ s.t. } \left. \sigma \right|_K = Id \text{ and } \sigma \text{ is an autotomorphism}$.\\
	Similarly, let $\tau = \Frob{L/K}{\mathfrak{p}} \in \Gal{L/K}$ so 
	$\tau: L \to L \text{ s.t. } \left. \tau \right|_K = Id \text{ and } \tau \text{ is an autotomorphism}$.
	
	As $M$ extends $L$, $\sigma$ being an automorphism of $M$ makes it an automorphism of $L$ as well.
	The restriction condition stays the same.
	% done with Djordjo
	% \cite[p.99]{AlgebraicNumberFields}
\end{proof}


\begin{property}
	$$
	\Gal{L/K} \cong \nicefrac{\Gal{M/K}}{\Gal{M/L}}
	$$
\end{property}
\begin{proof}
	Let $\sigma \in \Gal{M/K}$, i.e. $\sigma: M \to M \text{ s.t. } \left. \sigma \right|_K = Id \text{ and } \sigma \text{ is an autotomorphism}$.
	
	Let $\phi: \Gal{M/K} \to \Gal{L/K}$ be such that:
	$\phi(\sigma) = \left. \sigma \right|_L$.
	This is well defined as an automorphism of $M$ restricts to an automorphism of $L$ when $M$ extends $L$.
	
	It is trivial to check that $\phi$ is a homomorphism.
	
	The kernel of $\phi$ is clearly $\Gal{M/L}$.
	
	The image of $\phi$ is $\Gal{L/K}$ as every element of $\Gal{L/K}$ may be extended to $\Gal{M/K}$.
	
	Therefore, the property follows via the $1^{st}$ isomorphism theorem.
	% done with Djordjo
\end{proof}

\begin{property}
	We have: 
	$$
	\mathfrak{p} \text{ splits complitely in } L 
	\iff \Frob{L/K}{\mathfrak{P}} = 1
	$$
\end{property}
\cite[p.100]{AlgebraicNumberFields}


We will consider the following scheme:\\
\begin{tikzpicture}[text width=10cm, align=flush center, node distance=2cm]
%nodes
\node (M) {$L_1L_2$};
\node (supsetM)[right of=M, node distance=0.75cm] {$\supseteq$};
\node (PM)[right of=supsetM, node distance=0.5cm] {$\mathfrak{P}$};

\node (L1) [below of=M, left of=M] {$L_1$};
\node (supsetL1)[right of=L1, node distance=0.5cm] {$\supseteq$};
\node (PL1)[right of=supsetL1, node distance=0.5cm] {$\mathfrak{p}_1$};

\node (L2) [below of=M, right of=M] {$L_2$};
\node (supsetL2)[right of=L2, node distance=0.5cm] {$\supseteq$};
\node (PL2)[right of=supsetL2, node distance=0.5cm] {$\mathfrak{p}_2$};

\node (K) [below of=M, node distance = 4cm] {$K$};
\node (supsetK)[right of=K, node distance=0.5cm] {$\supseteq$};
\node (PK)[right of=supsetK, node distance=0.5cm] {$\mathfrak{p}$};


%links
\draw[-] (M) to node {} (L1);
\draw[-] (M) to node {} (L2);
\draw[-] (L1) to node {} (K);
\draw[-] (L2) to node {} (K);
\end{tikzpicture}



\begin{property}
	We have:
	$$
	\Frob{L_1L_2/K}{\mathfrak{P}}
	= \Frob{L_1/K}{\mathfrak{p}_1} \times \Frob{L_2/K}{\mathfrak{p}_2}
	$$
\end{property}
\cite[p.100]{AlgebraicNumberFields}

\begin{property}
	We have: 
	$$
	\mathfrak{p} \text{ splits complitely in } L_1L_2
	\iff
	\mathfrak{p} \text{ splits complitely in } L_1 \text{ and } L_2
	$$
\end{property}
\begin{proof}
	Combine the last two proposition.
	\cite[p.100]{AlgebraicNumberFields}
\end{proof}





\subsection{The Chebotarev's Density Theorem}
\subsubsection{Motivations}
If we look at the distribution of primes numbers modulo a number (15 in the next example), we get a table as follows:

Table mod 15:\\
0: \\
1: 31, 61, 151, 181, 211, 241, 271, 331, 421, \\
2: 2, 17, 47, 107, 137, 167, 197, 227, 257, 317, 347, 467, \\
3: 3, \\
4: 19, 79, 109, 139, 199, 229, 349, 379, 409, 439, 499, \\
5: 5, \\
6: \\
7: 7, 37, 67, 97, 127, 157, 277, 307, 337, 367, 397, 457, 487, \\
8: 23, 53, 83, 113, 173, 233, 263, 293, 353, 383, 443, \\
9: \\
10: \\
11: 11, 41, 71, 101, 131, 191, 251, 281, 311, 401, 431, 461, 491, \\
12: \\
13: 13, 43, 73, 103, 163, 193, 223, 283, 313, 373, 433, 463, \\
14: 29, 59, 89, 149, 179, 239, 269, 359, 389, 419, 449, 479, \\

%[table display to improve!!!]
It looks like there are classes of primes.
We would like to characterize this repartition: that is, decide if classes are finite or infinite, and quantify the repartitions.



\subsubsection{Notions of Density}
As discussed previously, we are interested in subsets of $\primes$ (the set of primes numbers).
Euler proved that there are infinitely many primes.
%cite Euler?
Therefore, there are two types of subsets of $\primes$: the ones that are infinite, and the finites ones.
For finite sets, we can characterise the size by just counting elements.
In fact, we will mainly be interested in sets that have infinitely many primes, and again, we would like a notion of size.

A suitable way would be to compare the subset with the set of all primes, and, say look at the proportions of primes included in the subset.

We call this the density, there are two rigorous ways to define it:
\begin{definition}[Natural density]
	We say that $S \subseteq \primes$ has natural density $\delta$ when:
	$$
	\lim_{x \to +\infty}
	\frac{ \# \{ p \in \primes | p \in S \}}
	{ \# \{ p \in \primes | p \in \primes \}} = \delta
	$$
\end{definition}
\begin{definition}[Analytic density or Dirichlet density]
	We say that $S \subseteq \primes$ has analytical (or Dirichlet) density $\delta$ when:
	$$
	\lim_{s \to 1^+}
	\left( \sum_{p \in S} \frac{1}{p^s} \right) 
	\left( \sum_{p \in \primes} \frac{1}{p} \right)^{-1} = \delta
	$$
\end{definition}

Note that the natural density may not exist.
However, when both exist, the two densities are the same.
%cite!



\subsubsection{Theorem}
One of the most important results that use Frobenian maps is probably the Chebotarev density theorem.
\begin{theorem}
	With $L/K$ an extension of Galois group $G=\Gal{L/K}$.\\
	Let $C$ be a conjugacy class in $G$.
	
	Then, the proportion of unramified primes ideals $\mathfrak{p}$ in $K$ that have Frobenius element $\Frob{L/K}{\mathfrak{p}}=C$ \footnote{When depending on a prime in the "lower" field, the Frobenius element is a conjugacy class to be well defined} is $\nicefrac{|C|}{|G|}$.
\end{theorem}
We see that Frobenius elements are in the heart of this theorem.

\paragraph{Proof}
[quick proof? actually, I couldn't find one, maybe just idea of proof?]

\paragraph{Example}
Extension with non-commutative Galois group (as commutative will be used in Dirichlet's example).
=> splitting field of $x^3-2$? (i.e. $\Q[\zeta_3, \curt{2}]:\Q$)



\subsection{The Dirichlet's Density Theorem}
\subsubsection{Theorem}
The most common application of Chebotarev density theorem is probably the Dirichlet's density theorem.
It states as follows:
\begin{theorem}
	Let $n \in \N^*$, $a \in \N$ such that $\gcd(a,n) = 1$. 
	If $S = \{ p \in \primes | p \equiv a \mod n \}$, then $S$ has density $\nicefrac{1}{\varphi(n)}$.
\end{theorem}

\subsubsection{Link with Chebotarev}
This is a direct application of Chebotarev's density theorem for the field extension $\Q[\gamma]:\Q$ where $\gamma$ is the $n^{th}$ root of unity (this is the cyclotomic field).

The Galois group then is cyclic, as this is a cyclotomic extension, and has order $\varphi(n)$ by definition (so $\Gal{\Q[\gamma]:\Q} \cong \left( \nicefrac{\Z}{n\Z} \right) \cong C_{\varphi(n)}$).

Primes ideals in $\Q$ are just primes numbers. And as the Galois group is cyclic, it is abelian, so all conjugacy classes have one element.
Thus, Chebotarev gives Dirichlet's density theorem in the particular case of cyclotomic extensions.
%[need more details?]

\subsubsection{Proof}
[Stand alone proof?]

\subsubsection{Example}
[Prove the case n=15 from the motivation subsection above.]







